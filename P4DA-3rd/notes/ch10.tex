\chapter{数据聚合与分组运算\label{ch10}}
\section{Data Aggregation}
\begin{table}
    \centering
    \caption{Optimized groupby methods}
    \label{tbl10.1}
    \begin{tabular}{ll}
        \hline
        Function name  & Description                                                                  \\
        \hline
        any, all       & Return True if any (one or more values) or all non-NA values are “truthy”    \\
        count          & Number of non-NA values                                                      \\
        cummin, cummax & Cumulative minimum and maximum of non-NA values                              \\
        cumsum         & Cumulative sum of non-NA values                                              \\
        cumprod        & Cumulative product of non-NA values                                          \\
        first, last    & First and last non-NA values                                                 \\
        mean           & Mean of non-NA values                                                        \\
        median         & Arithmetic median of non-NA values                                           \\
        min, max       & Minimum and maximum of non-NA values                                         \\
        nth            & Retrieve value that would appear at position n with the data in sorted order \\
        ohlc           & Compute four “open-high-low-close” statistics for time series-like data      \\
        prod           & Product of non-NA values                                                     \\
        quantile       & Compute sample quantile                                                      \\
        rank           & Ordinal ranks of non-NA values, like calling Series.rank                     \\
        size           & Compute group sizes, returning result as a Series                            \\
        sum            & Sum of non-NA values                                                         \\
        std, var       & Sample standard deviation and variance                                       \\
        \hline
    \end{tabular}
\end{table}
\section{apply:一般性的“拆分-应用-合并”}
如果传给 apply 的函数能够接受其他参数或关键字,则可以将这些内容放在函数名后面一并传入。传入的那个函数能做什么全由你说了算,它只需返回一个pandas对象或标量值即可。
\subsection{禁止分组键}
\subsection{分位数和封箱分析}
由 pd.cut(equal-length) 返回的 pd.Categorical 对象可直接传递到 groupby。