\chapter{The Basics of NumPy Arrays}
We'll cover a few categories of basic array manipulations here:
\begin{enumerate}
    \item \emph{Attributes of arrays}: Determining the size, shape, memory consumption, and data types of arrays
    \item \emph{Indexing of arrays}: Getting and setting the values of individual array elements
    \item \emph{Slicing of arrays}: Getting and setting smaller subarrays within a larger array
    \item \emph{Reshaping of arrays}: Changing the shape of a given array
    \item \emph{Joining and splitting of arrays}: Combining multiple arrays into one, and splitting one array into many
\end{enumerate}

\section{NumPy Array Attributes}
Each array has attributes including ndim (the number of dimensions), shape (the size of each dimension), size (the total size of the array), and dtype (the type of each element)

\section{Array Indexing: Accessing Single Elements}
\begin{itemize}
    \item In a one-dimensional array, the ith value (counting from zero) can be accessed by specifying the desired index in square brackets, just as with Python lists
    \item To index from the end of the array, you can use negative indices
    \item In a multidimensional array, items can be accessed using a comma-separated (row, column) tuple
    \item Keep in mind that, unlike Python lists, NumPy arrays have a fixed type. This means, for example, that \important{if you attempt to insert a floating-point value into an integer array, the value will be silently truncated}. Don't be caught unaware by this behavior!
\end{itemize}

\section{Array Slicing: Accessing Subarrays}
Just as we can use square brackets to access individual array elements, we can also use
them to access subarrays with the slice notation, marked by the colon (:) character.
The NumPy slicing syntax follows that of the standard Python list; to access a slice of
an array x, use this:

\verb|x[start:stop:step]|

If any of these are unspecified, they default to the values \verb|start=0, stop=<size of dimension>, step=1|.

\subsection{One-Dimensional Subarrays}
A potentially confusing case is when the step value is negative. In this case, the
defaults for start and stop are swapped. This becomes a convenient way to reverse
an array. 如果step大于0,则start要小于stop,反之要大于,因为每一步之后要进行step处理。
\subsection{Multidimensional Subarrays}
Multidimensional slices work in the same way, with multiple slices separated by commas.

One commonly needed routine is accessing single rows or columns of an array. This
can be done by combining indexing and slicing, using an empty slice marked by a
single colon (:).

In the case of row access, the empty slice can be omitted for a more compact syntax
\subsection{Subarrays as No-Copy Views}
\important{Unlike Python list slices, NumPy array slices are returned as \emph{views}\marginpar[views]{views} rather than \emph{copies} of the array data.}

Now if we modify this subarray, we'll see that the original array is changed!

Some users may find this surprising, but it can be advantageous: for example, when
working with large datasets, we can access and process pieces of these datasets
without the need to copy the underlying data buffer.
\subsection{Creating Copies of Arrays}

\section{Reshaping of Arrays}

Another useful type of operation is reshaping of arrays, which can be done with the
\verb|reshape| method. In most cases the reshape method will return a no-copy view of
the initial array.

A common reshaping operation is converting a one-dimensional array into a two-dimensional row or column matrix.

A convenient shorthand for this is to use \verb|np.newaxis| in the slicing syntax.

\section{Array Concatenation and Splitting}
\subsection{Concatenation of Arrays}

Concatenation, or joining of two arrays in NumPy, is primarily accomplished using
the routines \verb|np.concatenate|, \verb|np.vstack|, and \verb|np.hstack|. \verb|np.concatenate| takes a
tuple or list of arrays as its first argument.

For working with arrays of mixed dimensions, it can be clearer to use the np.vstack
(vertical stack) and np.hstack (horizontal stack) functions.


Similarly, for higher-dimensional arrays, np.dstack will stack arrays along the third
axis.

\subsection{Splitting of Arrays}
The opposite of concatenation is splitting, which is implemented by the functions
\verb|np.split|, \verb|np.hsplit|, and \verb|np.vsplit|. For each of these, we can pass a list of indices
giving the split points.

Notice that $N$ split points leads to $N + 1$ subarrays. The related functions np.hsplit
and np.vsplit are similar.
