\chapter{Handling Missing Data\label{Ch16}}
In this chapter, we will discuss some general considerations for missing data, look at
how Pandas chooses to represent it, and explore some built-in Pandas tools for handling missing data in Python.

\section{Trade-offs in Missing Data Conventions}
A number of approaches have been developed to track the presence of missing data in
a table or DataFrame. Generally, they revolve around one of two strategies: using a
mask that globally indicates missing values, or choosing a sentinel value(标签值) that indicates
a missing entry.

In the masking approach, the mask might be an entirely separate Boolean array, or it
might involve appropriation of one bit in the data representation to locally indicate
the null status of a value.

In the sentinel approach, the sentinel value could be some data-specific convention,
such as indicating a missing integer value with –9999 or some rare bit pattern, or it
could be a more global convention, such as indicating a missing floating-point value
with NaN (Not a Number), a special value that is part of the IEEE floating-point
specification.

Neither of these approaches is without trade-offs. Use of a separate mask array
requires allocation of an additional Boolean array, which adds overhead in both storage and computation. A sentinel value reduces the range of valid values that can be
represented, and may require extra (often nonoptimized) logic in CPU and GPU
arithmetic, because common special values like NaN are not available for all data
types.

\section{Missing Data in Pandas}
The way in which Pandas handles missing values is constrained by its reliance on the
NumPy package, which does not have a built-in notion of NA values for non-
floating-point data types.

Because of these constraints and trade-offs, Pandas has two “modes” of storing and
manipulating null values:
\begin{itemize}
    \item The default mode is to use a sentinel-based missing data scheme, with sentinel
          values NaN or None depending on the type of the data.
    \item Alternatively, you can opt in to using the nullable data types (dtypes) Pandas provides (discussed later in this chapter), which results in the creation an accompanying mask array to track missing entries. These missing entries are then
          presented to the user as the special pd.NA value.
\end{itemize}
\subsection*{None as a Sentinel Value}
For some data types, Pandas uses None as a sentinel value. None is a Python object,
which means that any array containing None must have \verb|dtype=object|—that is, it
must be a sequence of Python objects.

This \verb|dtype=object| means that the best common type representation NumPy could
infer for the contents of the array is that they are Python objects. \tips{The downside of
    using None in this way is that operations on the data will be done at the Python level,
    with much more overhead than the typically fast operations seen for arrays with
    native types.}

Further, because Python does not support arithmetic operations with None, aggregations like sum or min will generally lead to an error.

For this reason, Pandas does not use None as a sentinel in its numerical arrays.
\subsection*{NaN: Missing Numerical Data}

The other missing data sentinel, NaN is different; it is a special floating-point value
recognized by all systems that use the standard IEEE floating-point representation.

这和使用 None 的数组不同,使用 np.nan 的数组会被编译成C代码从而实现快速操作。Keep in mind that NaN is a bit like a data virus—it infects any other
object it touches. Regardless of the operation, the result of arithmetic with NaN will be another NaN.

This means that aggregates over the values are well defined (i.e., they don't result in
an error) but not always useful. NumPy does provide NaN-aware versions of aggregations that will ignore
these missing values.

\notes{
    The main downside of NaN is that it is specifically a floating-point value; there is no
    equivalent NaN value for integers, strings, or other types.
}

\subsection*{NaN and None in Pandas}
NaN and None both have their place, and Pandas is built to handle the two of them
nearly interchangeably, converting between them where appropriate.

For types that don't have an available sentinel value, Pandas automatically typecasts
when NA values are present. 比如:如果有一个整数组成的 Series,将 None 赋值给其中一个元素,那么Series的类型将转换为 float.

Notice that in addition to casting the integer array to floating point, Pandas automatically converts the None to a NaN value.

\autoref{Pandas handling of NAs by type} lists the upcasting conventions in Pandas when NA values are introduced.

\begin{table}
    \centering
    \caption{Pandas handling of NAs by type}
    \label{Pandas handling of NAs by type}
    \begin{tabular}{lll}
        \hline
        Typeclass & Conversion when storing NAs & NA sentinel value \\
        \hline
        floating  & No change                   & np.nan            \\
        object    & No change                   & None or np.nan    \\
        integer   & Cast to float64             & np.nan            \\
        boolean   & Cast to object              & None or np.nan    \\
        \hline
    \end{tabular}
\end{table}

Keep in mind that in Pandas, string data is always stored with an object dtype.

\section{Pandas Nullable Dtypes}
In early versions of Pandas, NaN and None as sentinel values were the only missing
data representations available. The primary difficulty this introduced was with regard
to the implicit type casting. \autoref{Pandas handling of NAs by type} 可以看出来,没有办法真正表示整数的缺失值,因为会被转成浮点型。

To address this difficulty, Pandas later added \textbf{nullable dtypes}\marginpar[nullable dtypes]{nullable dtypes}, which are distinguished
from regular dtypes by capitalization of their names (e.g., pd.Int32 versus np.int32).
For backward compatibility, these nullable dtypes are only used if specifically
requested.
\section{Operating on Null Values}
As we have seen, Pandas treats None, NaN, and NA as essentially interchangeable for
indicating missing or null values. To facilitate this convention, Pandas provides several methods for detecting, removing, and replacing null values in Pandas data structures. They are:

\begin{itemize}
    \item isnull: Generates a Boolean mask indicating missing values
    \item notnull: Opposite of isnull
    \item dropna: Returns a filtered version of the data
    \item fillna: Returns a copy of the data with missing values filled or imputed
\end{itemize}
\subsection*{Detecting Null Values}
Pandas data structures have two useful methods for detecting null data: isnull and
notnull. Either one will return a Boolean mask over the data.

\subsection*{Dropping Null Values}
In addition to these masking methods, there are the convenience methods dropna
(which removes NA values) and fillna (which fills in NA values).

For a DataFrame, there are more options. We cannot drop single values from a DataFrame; we can only drop entire rows or columns.

By default, dropna will drop all rows in which any null value is present. But this drops some good data as well; you might rather be interested in dropping
rows or columns with all NA values, or a majority of NA values. This can be specified
through the how or thresh parameters, which allow fine control of the number of
nulls to allow through.(可以这是方式或者阈值来删除行或者列,而不是默认的存在空值即删除)

The default is \verb|how='any'|\marginpar[how]{how}, such that any row or column containing a null value will be dropped. You can also specify \verb|how='all'|, which will only drop rows/columns that contain all null values.

For finer-grained control, the thresh\marginpar[thresh]{thresh} parameter lets you specify a \important{minimum number of non-null values for the row/column to be kept}.

Alternatively, you can drop NA values along a different axis.

\subsection*{Filling Null Values}

Sometimes rather than dropping NA values, you'd like to replace them with a valid
value. This value might be a single number like zero, or it might be some sort of
imputation or interpolation from the good values. You could do this in-place using
the isnull method as a mask, but because it is such a common operation Pandas
provides the fillna method, which returns a copy of the array with the null values
replaced.

We can fill NA entries with a single value, such as zero, can specify a forward fill to propagate the previous value forward, or specify a backward fill to propagate the next values backward.

In the case of a DataFrame, the options are similar, but we can also specify an axis
along which the fills should take place.


