\chapter{General Matplotlib Tips\label{Ch25}}
\section{用不用show()?如何显示图形}
如何显示你的图形,就取决于
具体的开发环境。Matplotlib 的最佳实践与你使用的开发环境有关。简单来说,就是有
三种开发环境,分别是脚本、IPython shell 和 IPython Notebook。

\subsection*{Plotting from a Script}
One thing to be aware of: the plt.show command should be used only once per
Python session, and is most often seen at the very end of the script. Multiple show
commands can lead to unpredictable backend-dependent behavior, and should
mostly be avoided.

\subsection*{Plotting from an IPython Shell}
\subsection*{Plotting from a Jupyter Notebook}
Plotting interactively within a Jupyter notebook can be done with the \verb|%matplotlib|
command, and works in a similar way to the IPython shell. You also have the option
of embedding graphics directly in the notebook, with two possible options:
\begin{itemize}
    \item \verb|%matplotlib inline| will lead to static images of your plot embedded in the
          notebook.
    \item \verb|%matplotlib notebook| will lead to interactive plots embedded within the note‐
          book.
\end{itemize}

\subsection*{Saving Figures to File}
One nice feature of Matplotlib is the ability to save figures in a wide variety of formats. Saving a figure can be done using the \verb|savefig|\marginpar[savefig]{savefig} command. In savefig, the file format is inferred from the extension of the given filename. In savefig, the file format is inferred from the extension of the given filename.
Depending on what backends you have installed, many different file formats are
available. The list of supported file types can be found for your system by using the
following method of the figure canvas object:

\verb|fig.canvas.get_supported_filetypes()|

\subsection*{Two Interfaces for the Price of One}
A potentially confusing feature of Matplotlib is its dual interfaces: a convenient
MATLAB-style state-based interface, and a more powerful object-oriented interface.

\subsubsection*{MATLAB-style Interface}
Matplotlib was originally conceived as a Python alternative for MATLAB users, and
much of its syntax reflects that fact. The MATLAB-style tools are contained in the
pyplot (plt) interface.

It is important to recognize that this interface is stateful: it keeps track of the “current”
figure and axes, which are where all plt commands are applied. You can get a reference to these using the plt.gcf (get current figure) and plt.gca (get current axes)
routines.

\important{
    While this stateful interface is fast and convenient for simple plots, it is easy to run
    into problems. For example, once the second panel is created, how can we go back
    and add something to the first?
}

\subsubsection*{Object-oriented interface}
Rather than depending on some
notion of an “active” figure or axes, in the object-oriented interface the plotting functions are methods of explicit Figure and Axes objects.

For simpler plots, the choice of which style to use is largely a matter of preference, but
the object-oriented approach can become a necessity as plots become more complicated