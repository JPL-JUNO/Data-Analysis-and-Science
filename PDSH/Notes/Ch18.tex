\chapter{Combining Datasets: concat and append\label{Ch18}}
Some of the most interesting studies of data come from combining different data
sources. These operations can involve anything from very straightforward concatenation of two different datasets to more complicated database-style joins and merges
that correctly handle any overlaps between the datasets.
\section{Recall: Concatenation of NumPy Arrays}
Concatenation of Series and DataFrame objects behaves similarly to concatenation
of NumPy arrays, which can be done via the \verb|np.concatenate|\marginpar[concatenate]{concatenate} function.

The first argument is a list or tuple of arrays to concatenate. Additionally, in the case
of multidimensional arrays, it takes an axis keyword that allows you to specify the
axis along which the result will be concatenated.


\section{Simple Concatenation with pd.concat}
The \verb|pd.concat|\marginpar[concat]{concat} function provides a similar syntax to np.concatenate but contains a
number of options.

pd.concat can be used for a simple concatenation of Series or DataFrame objects,
just as np.concatenate can be used for simple concatenations of arrays.

It's default behavior is to concatenate row-wise within the DataFrame (i.e., axis=0).
Like np.concatenate, pd.concat allows specification of an axis along which concatenation will take place.
\subsection*{Duplicate Indices}
One important difference between np.concatenate and pd.concat is that Pandas
concatenation \emph{preserves indices}, even if the result will have duplicate indices! While this is valid within DataFrames, the
outcome is often undesirable. pd.concat gives us a few ways to handle it.
\subsubsection*{Treating repeated indices as an error}

If you'd like to simply verify that the indices in the result of pd.concat do not overlap,
you can include the \verb|verify_integrity| flag. With this set to True, the concatenation
will raise an exception if there are duplicate indices.
\subsubsection*{Ignoring the index}
Sometimes the index itself does not matter, and you would prefer it to simply be
ignored. This option can be specified using the \verb|ignore_index| flag. With this set to
True, the concatenation will create a new integer index for the resulting DataFrame
\subsubsection*{Adding MultiIndex keys}
Another option is to use the keys option to specify a label for the data sources; the
result will be a hierarchically indexed series containing the data
\subsubsection*{Concatenation with Joins}
In practice, data from different sources might have different sets of column names, and pd.concat offers several options in this case.

The default behavior is to fill entries for which no data is available with NA values. To
change this, we can adjust the join parameter of the concat function. By default, the
join is a union of the input columns (\verb|join='outer'|), but we can change this to an
intersection of the columns using \verb|join='inner'|.

Another useful pattern is to use the reindex method before concatenation for finer
control over which columns are dropped.(这种效果似乎可以实现左连接或者右连接,因为参数 join 只能是内连接和外连接两种)
\subsection*{The append Method}
Because direct array concatenation is so common, Series and DataFrame objects
have an append method that can accomplish the same thing in fewer keystrokes.

Keep in mind that unlike the append and extend methods of Python lists, the append
method in Pandas does not modify the original object; instead \notes{it creates a new object with the combined data}. It also is not a very efficient method, because it involves creation of a new index and data buffer. Thus, if you plan to do multiple append operations, it is generally better to build a list of DataFrame objects and pass them all at once to the concat function.

\important{
    FutureWarning: The frame.append method is deprecated and will be removed from pandas in a future version. Use pandas.concat instead.
}