\chapter{Aggregation and Grouping\label{Ch20}}
A fundamental piece of many data analysis tasks is efficient summarization: computing aggregations like sum, mean, median, min, and max, in which a single number summarizes aspects of a potentially large dataset.
\section{Planets Data}
Here we will use the Planets dataset, available via the Seaborn package. It gives information on planets that astronomers have discovered around
other stars (known as extrasolar planets(系外星系), or exoplanets for short).
\section{Simple Aggregation in Pandas}
As with a one-dimensional NumPy array, for a Pandas Series the aggregates return a
single value.

For a DataFrame, by default the aggregates return results within each column. By specifying the axis argument, you can instead aggregate within each row.

Pandas Series and DataFrame objects include all of the common aggregates mentioned in \autoref{Ch07}; in addition, there is a convenience method, \verb|describe|\marginpar[describe]{describe}, that computes several common aggregates for each column and returns the result.

\autoref{Listing of Pandas aggregation methods} summarizes some other built-in Pandas aggregations.

\begin{table}
    \centering
    \caption{Listing of Pandas aggregation methods}
    \label{Listing of Pandas aggregation methods}
    \begin{tabular}{ll}
        \hline
        Aggregation  & Returns                         \\
        \hline
        count        & Total number of items           \\
        first, last  & First and last item             \\
        mean, median & Mean and median                 \\
        min, max     & Minimum and maximum             \\
        std, var     & Standard deviation and variance \\
        mad          & Mean absolute deviation         \\
        prod         & Product of all items            \\
        sum          & Sum of all items                \\
        \hline
    \end{tabular}
\end{table}
\section{groupby: Split, Apply, Combine}
Simple aggregations can give you a flavor of your dataset, but often we would prefer
to aggregate conditionally on some label or index: this is implemented in the so-called groupby operation.
\subsection*{Split, Apply, Combine}
A canonical example of this split-apply-combine operation, where the “apply” is a summation aggregation, is illustrated \autoref{A visual representation of a groupby operation}.

\autoref{A visual representation of a groupby operation} shows what the groupby operation accomplishes:
\begin{itemize}
    \item The split step involves breaking up and grouping a DataFrame depending on the value of the specified key.
    \item The apply step involves computing some function, usually an aggregate, transformation, or filtering, within the individual groups.
    \item The combine step merges the results of these operations into an output array.
\end{itemize}

\figures{A visual representation of a groupby operation}

The power of the groupby is that it
abstracts away these steps: the user need not think about how the computation is
done under the hood, but rather can think about the operation as a whole.

The most basic split-apply-combine operation can be computed with the groupby
method of the DataFrame, passing the name of the desired key column.

Notice that what is returned is a DataFrameGroupBy object, not a set of DataFrame
objects. This object is where the magic is: you can think of it as a special view of the
DataFrame, which is poised to dig into the groups but does no actual computation
until the aggregation is applied.
\subsection*{The GroupBy Object}
The GroupBy object is a flexible abstraction: in many ways, it can be treated as simply
a collection of DataFrames, though it is doing more sophisticated things under the
hood.

Perhaps the most important operations made available by a GroupBy are aggregate,
filter, transform, and apply.
\subsubsection*{Column indexing}
The GroupBy object supports column indexing in the same way as the DataFrame, and returns a modified GroupBy object.

\subsubsection*{Iteration over groups}

The GroupBy object supports direct iteration over the groups, returning each group as
a Series or DataFrame. This can be useful for manual inspection of groups for the sake of debugging, but it is
often much faster to use the built-in apply functionality
\subsubsection*{Dispatch methods}
Through some Python class magic, any method not explicitly implemented by the
GroupBy object will be passed through and called on the groups, whether they are
DataFrame or Series objects.

Notice that these dispatch methods are applied to each individual group, and the
results are then combined within GroupBy and returned. Again, any valid DataFrame/Series method can be called in a similar manner on the corresponding GroupBy
object.
\subsection*{Aggregate, Filter, Transform, Apply}
GroupBy objects have aggregate, fil
ter, transform, and apply methods that efficiently implement a variety of useful
operations before combining the grouped data.
\subsubsection*{Aggregation}
You're now familiar with GroupBy aggregations with sum, median, and the like, but the
aggregate method allows for even more flexibility. It can take a string, a function, or
a list thereof, and compute all the aggregates at once.

Another common pattern is to pass a dictionary mapping column names to operations to be applied on that column.

\subsubsection*{Filtering}
A filtering operation allows you to drop data based on the group properties. For
example, we might want to keep all groups in which the standard deviation is larger
than some critical value.

The filter function should \important{return a Boolean value} specifying whether the group passes
the filtering.
\subsubsection*{Transformation}
While aggregation must return a reduced version of the data, transformation can
return some transformed version of the full data to recombine. For such a transformation, the output is the same shape as the input. A common example is to center the
data by subtracting the group-wise mean
\subsubsection*{The apply method}
The apply method lets you apply an arbitrary function to the group results. The
function should take a DataFrame and returns either a Pandas object (e.g., DataFrame,
Series) or a scalar; the behavior of the combine step will be tailored to the type of
output returned.

apply within a GroupBy is flexible: the only criterion is that the function takes a
DataFrame and returns a Pandas object or scalar.
\subsection*{Specifying the Split Key}
In the simple examples presented before, we split the DataFrame on a single column
name. This is just one of many options by which the groups can be defined.
\subsubsection*{A list, array, series, or index providing the grouping keys}
The key can be any series or list with a length matching that of the DataFrame.


\subsubsection*{A dictionary or series mapping index to group}
Another method is to provide a dictionary that \important{maps index values to the group keys}.
\subsubsection*{Any Python function}
Similar to mapping, you can pass any Python function that will input the index value
and output the group.

\subsubsection*{A list of valid keys}
Further, any of the preceding key choices can be combined to group on a multi-index.
\subsection*{Grouping Example}