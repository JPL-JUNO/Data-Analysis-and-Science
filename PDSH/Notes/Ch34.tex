\chapter{Customizing Matplotlib: Configurations and Stylesheets\label{Ch34}}
While many of the topics covered in previous chapters involve adjusting the style of
plot elements one by one, Matplotlib also offers mechanisms to adjust the overall
style of a chart all at once. In this chapter we'll walk through some of Matplotlib's runtime configuration (\textbf{rc}) options, and take a look at the stylesheets feature, which
contains some nice sets of default configurations.

\section{Plot Customization by Hand}

\section{Changing the Defaults: rcParams}
Each time Matplotlib loads, it defines a runtime configuration containing the default
styles for every plot element you create. This configuration can be adjusted at any
time using the \verb|plt.rc|\marginpar[plt.rc]{plt.rc} convenience routine.

Optionally, these settings can be saved in a \textbf{.matplotlibrc} file.

\section{Stylesheets}
A newer mechanism for adjusting overall chart styles is via Matplotlib's style module, which includes a number of default stylesheets, as well as the ability to create and
package your own styles. These stylesheets are formatted similarly to the \textbf{.matplotlibrc}
files mentioned earlier, but must be named with a \textbf{.mplstyle} extension.
Even if you don't go as far as creating your own style, you may find what you're looking for in the built-in stylesheets. \verb|plt.style.available| contains a list of the available styles.

The standard way to switch to a stylesheet is to call \verb|style.use|.

\important{But keep in mind that this will change the style for the rest of the Python session!
    Alternatively, you can use the style context manager, which sets a style temporarily.}

\subsection*{Default Style}
Matplotlib's default style was updated in the version 2.0 release.

\subsection*{FiveThirtyEight Style}
The fivethirtyeight style mimics the graphics found on the popular \href{https://fivethirtyeight.com/}{FiveThirtyEight website}. It is typified by bold colors, thick lines, and transparent axes.

\subsection*{ggplot Style}
The ggplot package in the R language is a popular visualization tool among data scientists. Matplotlib's ggplot style mimics the default styles from that package.

\subsection*{Bayesian Methods for Hackers Style}
There is a neat short online book called \href{https://dataorigami.net/Probabilistic-Programming-and-Bayesian-Methods-for-Hackers/}{Probabilistic Programming and Bayesian Methods for Hackers} by Cameron Davidson-Pilon that features figures created with
Matplotlib, and uses a nice set of rc parameters to create a consistent and visually
appealing style throughout the book. This style is reproduced in the bmh stylesheet.

\subsection*{Dark Background Style}
For figures used within presentations, it is often useful to have a dark rather than light
background. The \verb|dark_background| style provides this.

\subsection*{Grayscale Style}
You might find yourself preparing figures for a print publication that does not accept
color figures. For this, the grayscale style can be useful.

\subsection*{Seaborn Style}
Matplotlib also has several stylesheets inspired by the Seaborn library. I've found these settings to be very nice, and tend to use
them as defaults in my own data exploration