\chapter{Density and Contour Plots\label{Ch28}}
Sometimes it is useful to display three-dimensional data in two dimensions using
contours or color-coded regions. There are three Matplotlib functions that can be
helpful for this task: plt.contour\marginpar[plt.contour]{plt.contour} for contour plots, plt.contourf\marginpar[plt.contourf]{plt.contourf} for filled contour
plots, and plt.imshow\marginpar[plt.imshow]{plt.imshow} for showing images.

\section{Visualizing a Three-Dimensional Function}
A contour plot can be created with the plt.contour function. It takes three arguments: a grid of x values, a grid of y values, and a grid of z values. The x and y values
represent positions on the plot, and the z values will be represented by the contour
levels. Perhaps the most straightforward way to prepare such data is to use the
np.meshgrid\marginpar[np.meshgrid]{np.meshgrid} function, which builds two-dimensional grids from one-dimensional
arrays.

Notice that when a single color is used, negative values are represented by dashed
lines and positive values by solid lines. Alternatively, the lines can be color-coded by
specifying a colormap with the cmap\marginpar[cmap]{cmap} argument.

Here we chose the RdGy (short for Red–Gray) colormap, which is a good choice for
divergent data: (i.e., data with positive and negative variation around zero).

Our plot is looking nicer, but the spaces between the lines may be a bit distracting.
We can change this by switching to a filled contour plot using the plt.contourf
function, which uses largely the same syntax as plt.contour.

Additionally, we'll add a plt.colorbar command, which creates an additional axis
with labeled color information for the plot (see \autoref{Visualizing three-dimensional data with filled contours}).

\figures{Visualizing three-dimensional data with filled contours}

One potential issue with this plot is that it is a bit splotchy: the color steps are discrete
rather than continuous, which is not always what is desired. This could be remedied
by setting the number of contours to a very high number, but this results in a rather
inefficient plot: Matplotlib must render a new polygon for each step in the level. A
better way to generate a smooth representation is to use the plt.imshow function,
which offers the interpolation argument to generate a smooth two-dimensional
representation of the data.

There are a few potential gotchas with plt.imshow, however(\important{以下这几条都很重要}):
\begin{itemize}
    \item It doesn't accept an x and y grid, so you must manually specify the extent [xmin,
                  xmax, ymin, ymax] of the image on the plot.
    \item By default it follows the standard image array definition where the origin is in the
          upper left, not in the lower left as in most contour plots. This must be changed
          when showing gridded data.
    \item It will automatically adjust the axis aspect ratio to match the input data; this can
          be changed with the aspect argument.
\end{itemize}

Finally, it can sometimes be useful to combine contour plots and image plots. For
example, here we'll use a partially transparent background image (with transparency
set via the alpha parameter) and overplot contours with labels on the contours themselves, using the plt.clabel function(see \autoref{Labeled contours on top of an image}):


\figures{Labeled contours on top of an image}

\begin{pyc}
    contours = plt.contour(X, Y, Z, 3, colors='black')
    plt.clabel(contours, inline=True, fontsize=8)
    plt.imshow(Z, extent=[0, 5, 0, 5], origin='lower', aspect='equal',
    cmap='RdGy', alpha=.5, interpolation='gaussian')
    plt.colorbar()
\end{pyc}

The combination of these three functions—plt.contour, plt.contourf, and
plt.imshow—gives nearly limitless possibilities for displaying this sort of three-
dimensional data within a two-dimensional plot.

\section{Histograms, Binnings, and Density}
A simple histogram can be a great first step in understanding a dataset.

The plt.hist docstring has more information on other available customization
options. I find this combination of \verb|histtype='stepfilled'| along with some transparency alpha to be helpful when comparing histograms of several distributions(see \autoref{Overplotting multiple histograms}).

\figures{Overplotting multiple histograms}

If you are interested in computing, but not displaying, the histogram (that is, counting the number of points in a given bin), you can use the \verb|np.histogram|\marginpar[np.histogram]{np.histogram} function.

\section{Two-Dimensional Histograms and Binnings}
Just as we create histograms in one dimension by dividing the number line into bins,
we can also create histograms in two dimensions by dividing points among two-dimensional bins.

\subsection*{plt.hist2d: Two-Dimensional Histogram}
One straightforward way to plot a two-dimensional histogram is to use Matplotlib's
\verb|plt.hist2d|\marginpar[plt.hist2d]{plt.hist2d} function.

Just as plt.hist has a counterpart in np.histogram, plt.hist2d has a counterpart in
np.histogram2d\marginpar[np.histogram2d]{np.histogram2d}.

\subsection*{plt.hexbin: Hexagonal Binnings}
The two-dimensional histogram creates a tesselation of squares across the axes.
Another natural shape for such a tesselation is the regular hexagon. For this purpose,
Matplotlib provides the plt.hexbin routine, which represents a two-dimensional
dataset binned within a grid of hexagons.

plt.hexbin has a number of additional options, including the ability to specify
weights for each point and to change the output in each bin to any NumPy aggregate
(mean of weights, standard deviation of weights, etc.).

\subsection*{Kernel Density Estimation}
Another common method for estimating and representing densities in multiple
dimensions is \textbf{kernel density estimation}\marginpar[kernel density estimation]{kernel density estimation} (KDE). Now I'll simply mention that KDE can be thought of as a way to
“smear out” the points in space and add up the result to obtain a smooth function.

KDE has a smoothing length that effectively slides the knob between detail and
smoothness (one example of the ubiquitous bias–variance trade-off).