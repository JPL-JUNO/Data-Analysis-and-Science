\chapter{Three-Dimensional Plotting in Matplotlib\label{Ch35}}
Three-
dimensional plots are enabled by importing the mplot3d toolkit, included with the
main Matplotlib installation.

Once this submodule is imported, a three-dimensional axes can be created by passing
the keyword \verb|projection='3d'| to any of the normal axes creation routines.

Three-dimensional plotting is one of the functionalities that
benefits immensely from viewing figures interactively rather than statically, in the
notebook; recall that to use interactive figures, you can use \verb|%matplotlib notebook|
rather than \verb|%matplotlib inline| when running this code.

\section{Three-Dimensional Points and Lines}
The most basic three-dimensional plot is a line or collection of scatter plots created
from sets of $(x, y, z)$ triples. In analogy with the more common two-dimensional plots
discussed earlier, these can be created using the \verb|ax.plot3D| and \verb|ax.scatter3D| functions. The call signature for these is nearly identical to that of their two-dimensional
counterparts.

\section{Three-Dimensional Contour Plots}
mplot3d contains tools to
create three-dimensional relief(晕渲) plots using the same inputs. Like \verb|ax.contour|, \verb|ax.contour3D| requires all the input data to be in the form of two-dimensional regular grids,
with the z data evaluated at each point.

Sometimes the default viewing angle is not optimal, in which case we can use the
\verb|view_init| method to set the elevation(角度) and azimuthal angles(方向角).

\section{Wireframes(线框图) and Surface Plots}
Two other types of three-dimensional plots that work on gridded data are wireframes
and surface plots. These take a grid of values and project it onto the specified three-
dimensional surface, and can make the resulting three-dimensional forms quite easy
to visualize.

A surface plot is like a wireframe plot, but each face of the wireframe is a filled polygon. Adding a colormap to the filled polygons can aid perception of the topology of
the surface being visualized.

Though the grid of values for a surface plot needs to be two-dimensional, it need not
be rectilinear(直角坐标系).

\section{Surface Triangulations}
For some applications, the evenly sampled grids required by the preceding routines
are too restrictive. In these situations, triangulation-based plots can come in handy.
What if rather than an even draw from a Cartesian or a polar grid, we instead have a
set of random draws?


The function that will help us in this case is
\verb|ax.plot_trisurf|, which creates a surface by first finding a set of triangles formed
between adjacent points.


