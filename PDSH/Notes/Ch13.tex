\chapter{Introducing Pandas Objects\label{Ch13}}

At a very basic level, Pandas objects can be thought of as enhanced versions of
NumPy structured arrays in which the rows and columns are identified with labels
rather than simple integer indices. As we will see during the course of this chapter,
Pandas provides a host of useful tools, methods, and functionality on top of the basic
data structures, but nearly everything that follows will require an understanding of
what these structures are. Thus, before we go any further, let's take a look at these
three fundamental Pandas data structures: the \verb|Series|\marginpar[Series]{Series}, \verb|DataFrame|\marginpar[DataFrame]{DataFrame}, and \verb|Index|\marginpar[Index]{Index}.

\section{The Pandas Series Object}
A Pandas Series is a one-dimensional array of indexed data. It can be created from a
list or array as follows.

The Series combines a sequence of values with an explicit sequence of indices,
which we can access with the \verb|values|\marginpar[values]{values} and \verb|index|\marginpar[index]{index} attributes. The values are simply a
familiar NumPy array.

The index is an array-like object of type pd.Index.

Like with a NumPy array, data can be accessed by the associated index via the familiar
Python square-bracket notation.
\subsection*{Series as Generalized NumPy Array}
From what we've seen so far, the Series object may appear to be basically interchangeable with a one-dimensional NumPy array. The essential difference is that
while the NumPy array has an implicitly defined integer index used to access the values, the Pandas Series has an explicitly defined index associated with the values.

This explicit index definition gives the Series object additional capabilities. For
example, the index need not be an integer, but can consist of values of any desired
type.

\subsection*{Series as Specialized Dictionary}
In this way, you can think of a Pandas Series a bit like a specialization of a Python
dictionary. A dictionary is a structure that maps arbitrary keys to a set of arbitrary
values, and a Series is a structure that maps typed keys to a set of typed values. This
typing is important: just as the type-specific compiled code behind a NumPy array
makes it more efficient than a Python list for certain operations, the type information
of a Pandas Series makes it more efficient than Python dictionaries for certain
operations.

The Series-as-dictionary analogy can be made even more clear by constructing a
Series object directly from a Python dictionary.

Unlike a dictionary, though, the Series also supports array-style operations such as
slicing.

\subsection*{Constructing Series Objects}
We've already seen a few ways of constructing a Pandas Series from scratch. All of
them are some version of the following:

\verb|pd.Series(data, index=index)|

where index is an optional argument, and data can be one of many entities.

\section{The Pandas DataFrame Object}
Like the Series object
discussed in the previous section, the DataFrame can be thought of either as a generalization of a NumPy array, or as a specialization of a Python dictionary.

\subsection*{DataFrame as Generalized NumPy Array}

Like the Series object, the DataFrame has an index attribute that gives access to the
index labels.

Additionally, the DataFrame has a columns attribute, which is an Index object holding
the column labels.

\subsection*{DataFrame as Specialized Dictionary}

Similarly, we can also think of a DataFrame as a specialization of a dictionary. Where
a dictionary maps a key to a value, a DataFrame maps a column name to a Series of
column data.

Notice the potential point of confusion here: in a two-dimensional NumPy array,
\verb|data[0]| will return the first row. For a DataFrame, \verb|data['col0']| will return the first
column. Because of this, it is probably better to think about DataFrames as generalized
dictionaries rather than generalized arrays, though both ways of looking at the situation can be useful.

\subsection*{Constructing DataFrame Objects}
A Pandas DataFrame can be constructed in a variety of ways. Here we'll explore several examples.
\subsubsection*{From a single Series object}
A DataFrame is a collection of Series objects, and a single-column DataFrame can be
constructed from a single Series.

\subsection*{From a list of dicts}

Any list of dictionaries can be made into a DataFrame. Even if some keys in the dictionary are missing, Pandas will fill them in with NaN
values.

\subsection*{From a dictionary of Series objects}
A DataFrame can be constructed from a dictionary of Series
objects as well.

\subsection*{From a two-dimensional NumPy array}
Given a two-dimensional array of data, we can create a DataFrame with any specified
column and index names.

\subsection*{From a NumPy structured array}

A Pandas DataFrame operates much like
a structured array, and can be created directly from one.


\section{The Pandas Index Object}
As you've seen, the Series and DataFrame objects both contain an explicit index that
lets you reference and modify data. This Index object is an interesting structure in
itself, and it can be thought of either as an immutable array or as an ordered set (technically a multiset, as Index objects may contain repeated values).

\subsection*{Index as Immutable Array}
The Index in many ways operates like an array. For example, we can use standard
Python indexing notation to retrieve values or slices.

Index objects also have many of the attributes familiar from NumPy arrays.

One difference between Index objects and NumPy arrays is that the indices are
immutable—that is, they cannot be modified via the normal means. This immutability makes it safer to share indices between multiple DataFrames and
arrays, without the potential for side effects from inadvertent index modification.

\subsection*{Index as Ordered Set}
Pandas objects are designed to facilitate operations such as joins across datasets,
which depend on many aspects of set arithmetic. The Index object follows many of
the conventions used by Python's built-in set data structure, so that unions, intersections, differences, and other combinations can be computed in a familiar way.

