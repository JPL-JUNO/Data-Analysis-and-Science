\chapter{Fancy Indexing}
In this chapter, we'll look at another style of array indexing, known as fancy or
vectorized indexing, in which we pass arrays of indices in place of single scalars. This
allows us to very quickly access and modify complicated subsets of an array's values.

\section{Exploring Fancy Indexing}
Fancy Indexing 在概念上非常简单,它意味着传递一个索引数组来一次性获得多个数组元素。

When using arrays of indices, the shape of the result reflects the shape of the \emph{index arrays} rather than the shape of the \emph{array being indexed}.

Fancy indexing also works in multiple dimensions. The pairing of indices in fancy indexing follows all the broadcasting
rules that were mentioned in \autoref{Ch08}.

这里特别需要记住的是,花哨的索引返回的值反映的是广播后的索引数组的形状,而不是
被索引的数组的形状。
\section{Combined Indexing}
We can combine fancy and simple indices, combine fancy indexing with slicing, combine fancy indexing with masking.

\section{Modifying Values with Fancy Indexing}
Just as fancy indexing can be used to access parts of an array, it can also be used to
modify parts of an array.

Notice, though, that repeated indices with these operations can cause some potentially unexpected results.

\begin{pyc}
    x = np.array([6., 0., 0., 0., 0., 0., 0., 0., 0., 0.])
    i = [2, 3, 3, 4, 4, 4]
    x[i] += 1
    x # array([6., 0., 1., 1., 1., 0., 0., 0., 0., 0.])
\end{pyc}
分析这段代码,需要将\verb|x[i] += 1|改写成\verb|x[i] = x[i] + 1|,这样先计算\verb|x[i] + 1|,然后将结果赋值给\verb|x[i]|,执行中间过程如下:

\begin{pyc}
    x = np.zeros(10)
    x[[0, 0]] = [4, 6]
    temp = x[i] + 1
    print(temp) # [1. 1. 1. 1. 1. 1.]
    x[i] = temp
    x[i] # array([1., 1., 1., 1., 1., 1.])
\end{pyc}

这里的重复赋值会因为重复的索引而出现替换,因此仅会出现最后一个索引的位置被修改。

So what if you want the other behavior where the operation is repeated? For this, you
can use the at method of ufuncs.

The at method does an in-place\marginpar[in-place]{in-place} application of the given operator at the specified indices with the specified value . Another method that is similar in
spirit is the reduceat method of ufuncs, which you can read about in the NumPy
documentation.



