\chapter{Vectorized String Operations\label{Ch22}}
One strength of Python is its relative ease in handling and manipulating string data.
Pandas builds on this and provides a comprehensive set of vectorized string operations
that are an important part of the type of munging required when working with (read:
cleaning up) real-world data.


\section{Introducing Pandas String Operations}
\important{Pandas includes features to address both this need for vectorized string operations as
    well as the need for correctly handling missing data via the str\marginpar[str]{str} attribute of Pandas
    Series and Index objects containing strings.}(通过 \verb|str| 属性来实现字符串的向量化操作)
\section{Tables of Pandas String Methods}
If you have a good understanding of string manipulation in Python, most of the Pandas string syntax is intuitive enough that it's probably sufficient to just list the available methods.

\subsection*{Methods Similar to Python String Methods}
Nearly all of Python's built-in string methods are mirrored by a Pandas vectorized
string method. The following Pandas str methods mirror Python string methods:
\begin{table}[H]
    \centering
    \begin{tabular}{lllll}
        \hline
        len        & lower      & translate & islower & ljust    \\
        upper      & startswith & isupper   & rjust   & find     \\
        endswith   & isnumeric  & center    & rfind   & isalnum  \\
        isdecimal  & zfill      & index     & isalpha & split    \\
        strip      & rindex     & isdigit   & rsplit  & rstrip   \\
        capitalize & isspace    & partition & lstrip  & swapcase \\
        \hline
    \end{tabular}
\end{table}

Notice that these have various return values. Some, like lower, return a series of
strings. But some others return numbers or Boolean values. Still others return lists or other compound values for each element.

\subsection*{Methods Using Regular Expressions}
In addition, there are several methods that accept regular expressions (regexps) to
examine the content of each string element, and follow some of the API conventions
of Python's built-in re module (see \autoref{tab22-1}).

\begin{table}
    \centering
    \caption{Mapping between Pandas methods and functions in Python's re module}
    \label{tab22-1}
    \begin{tabular}{ll}
        \hline
        Method   & Description                                                          \\
        \hline
        match    & Calls re.match on each element, returning a Boolean.                 \\
        extract  & Calls re.match on each element, returning matched groups as strings. \\
        findall  & Calls re.findall on each element                                     \\
        replace  & Replaces occurrences of pattern with some other string               \\
        contains & Calls re.search on each element, returning a boolean                 \\
        count    & Counts occurrences of pattern                                        \\
        split    & Equivalent to str.split, but accepts regexps                         \\
        rsplit   & Equivalent to str.rsplit, but accepts regexps                        \\
        \hline
    \end{tabular}
\end{table}

\subsection*{Miscellaneous Methods}
Finally, \autoref{tab22-2} lists miscellaneous methods that enable other convenient
operations.

\begin{table}

    \centering
    \caption{Other Pandas string methods}
    \label{tab22-2}
    \begin{tabular}{ll}
        \hline
        Method               & Description                                                           \\
        \hline
        \verb|get|           & Indexes each element                                                  \\
        \verb|slice|         & Slices each element                                                   \\
        \verb|slice_replace| & Replaces slice in each element with the passed value                  \\
        \verb|cat|           & Concatenates strings                                                  \\
        \verb|repeat|        & Repeats values                                                        \\
        \verb|normalize|     & Returns Unicode form of strings                                       \\
        \verb|pad|           & Adds whitespace to left, right, or both sides of strings              \\
        \verb|wrap|          & Splits long strings into lines with length less than a given width    \\
        \verb|join|          & Joins strings in each element of the Series with the passed separator \\
        \verb|get_dummies|   & Extracts dummy variables as a DataFrame                               \\
        \hline
    \end{tabular}
\end{table}
\subsubsection*{Vectorized item access and slicing}
The get and slice operations, in particular, enable vectorized element access from
each array. This behavior is also available through Python's normal
indexing syntax.

Indexing via \verb|df.str.get(i)| and \verb|df.str[i]| are likewise similar.

These indexing methods also let you access elements of arrays returned by split. For
example, to extract the last name of each entry, combine split with str indexing.

\subsubsection*{Indicator variables}
Another method that requires a bit of extra explanation is the \verb|get_dummies| method.
This is useful when your data has a column containing some sort of coded indicator.

\section{Going Further with Recipes}
\subsection*{Going Further with Recipes}
Extracting full ingredient lists from each recipe would be an important piece of the
task; unfortunately, the wide variety of formats used makes this a relatively time-consuming process. This points to the truism that in data science, cleaning and
munging of real-world data often comprises the majority of the work.