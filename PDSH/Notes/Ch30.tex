\chapter{Customizing Colorbars\label{Ch30}}
Plot legends identify discrete labels of discrete points. For continuous labels based on
the color of points, lines, or regions, a labeled colorbar can be a great tool. In Matplotlib, a colorbar is drawn as a separate axes that can provide a key for the meaning
of colors in a plot.

As we have seen several times already, the simplest colorbar can be created with the
plt.colorbar\marginpar[plt.colorbar]{plt.colorbar} function.

\section{Customizing Colorbars}
The colormap can be specified using the cmap argument to the plotting function that
is creating the visualization. The names of available colormaps are in the \verb|plt.cm| namespace.

\subsection*{Choosing the Colormap}
Broadly, you should be aware of three different categories of colormaps:
\begin{itemize}
    \item  \textbf{Sequential colormaps}(顺序配色方案): These are made up of one continuous sequence of colors (e.g., binary or
          viridis).
    \item  \textbf{Divergent colormaps}(互逆配色方法): These usually contain two distinct colors, which show positive and negative deviations from a mean (e.g., RdBu or PuOr).
    \item  \textbf{Qualitative colormaps}(定性配色方法): These mix colors with no particular sequence (e.g., rainbow or jet).
\end{itemize}

The jet colormap, which was the default in Matplotlib prior to version 2.0, is an
example of a qualitative colormap. Its status as the default was quite unfortunate,
because qualitative maps are often a poor choice for representing quantitative data.
\tips{Among the problems is the fact that qualitative maps usually do not display any uniform progression in brightness as the scale increases.}
We can see this by converting the jet colorbar into black and white (see \autoref{The jet colormap and its uneven luminance scale})

\figures{The jet colormap and its uneven luminance scale}

Notice the bright stripes in the grayscale image. Even in full color, this uneven brightness means that the eye will be drawn to certain portions of the color range, which
will potentially emphasize unimportant parts of the dataset. It’s better to use a colormap such as viridis (the default as of Matplotlib 2.0), which is specifically constructed to have an even brightness variation across the range; thus, it not only plays well
with our color perception, but also will translate well to grayscale printing.

For other situations, such as showing positive and negative deviations from some
mean, dual-color colorbars such as RdBu (Red–Blue) are helpful. However, as you can
see in \autoref{The RdBu colormap and its luminance}, it’s important to note that the positive/negative information will be
lost upon translation to grayscale!

\figures{The RdBu colormap and its luminance}
更多关于绘图的可以参考,\href{https://journals.plos.org/ploscompbiol/article?id=10.1371/journal.pcbi.1003833}{Ten Simple Rules for Better Figures}.
\subsection*{Color Limits and Extensions}
Matplotlib allows for a large range of colorbar customization. The colorbar itself is
simply an instance of plt.Axes, so all of the axes and tick formatting tricks we’ve seen
so far are applicable. The colorbar has some interesting flexibility: for example, we
can narrow the color limits and indicate the out-of-bounds values with a triangular
arrow at the top and bottom by setting the extend property.

\subsection*{Discrete Colorbars}

Colormaps are by default continuous, but sometimes you’d like to represent discrete
values. The easiest way to do this is to use the \verb|plt.cm.get_cmap|\footnote{MatplotlibDeprecationWarning: The get\_cmap function was deprecated in Matplotlib 3.7 and will be removed two minor releases later. Use ``matplotlib.colormaps[name]`` or ``matplotlib.colormaps.get\_cmap(obj)`` instead.} function and pass the
name of a suitable colormap along with the number of desired bins.

\section{Example: Handwritten Digits}
Because each digit is defined by the hue of its 64 pixels, we can consider each digit to
be a point lying in 64-dimensional space: each dimension represents the brightness of
one pixel. Visualizing such high-dimensional data can be difficult, but one way to
approach this task is to use a dimensionality reduction technique such as manifold
learning to reduce the dimensionality of the data while maintaining the relationships
of interest. Dimensionality reduction is an example of unsupervised machine learning, and we will discuss it in more detail in \autoref{Ch37}.