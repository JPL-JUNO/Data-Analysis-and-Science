\chapter{High-Performance Pandas: eval and query\label{Ch24}}
As we've already seen in previous chapters, the power of the PyData stack is built
upon the ability of NumPy and Pandas to push basic operations into lower-level compiled code via an intuitive higher-level syntax: examples are vectorized/broadcasted
operations in NumPy, and grouping-type operations in Pandas. While these abstractions are efficient and effective for many common use cases, they often rely on the
creation of temporary intermediate objects, which can cause undue overhead in computational time and memory use.

To address this, Pandas includes some methods that allow you to directly access C-speed operations without costly allocation of intermediate arrays: \verb|eval|\marginpar[eval]{eval} and \verb|query|\marginpar[query]{query},
which rely on the \href{https://github.com/pydata/numexpr}{NumExpr package}.

\section{Motivating query and eval: Compound Expressions}
We've seen previously that NumPy and Pandas support fast vectorized operations. This is much faster than doing the addition via a Python loop or comprehension. But this abstraction can become less efficient when computing compound expressions(复合代数式). In other words, \notes{every intermediate step is explicitly allocated in memory.} The NumExpr library gives you the ability to compute this type of compound
expression element by element, without the need to allocate full intermediate arrays. The \href{https://numexpr.readthedocs.io/en/latest/}{NumExpr documentation} has more details, but for the time being it is sufficient
to say that the library accepts a string giving the NumPy-style expression you'd like to
compute.

NumExpr evaluates the expression in a way that avoids temporary arrays where possible, and thus can be much more efficient than NumPy,
especially for long sequences of computations on large arrays. The Pandas eval and
query tools that we will discuss here are conceptually similar, and are essentially
Pandas-specific wrappers of NumExpr functionality.

\section{pandas.eval for Efficient Operations}
The eval function in Pandas uses string expressions to efficiently compute operations
on DataFrame objects.

\verb|pd.eval| supports a wide range of operations. Here's a summary of the operations pd.eval supports:
\begin{itemize}
    \item \textbf{Arithmetic operators}: pd.eval supports all arithmetic operators.
    \item \textbf{Comparison operators}: pd.eval supports all comparison operators, including chained expressions.
    \item \textbf{Bitwise operators}: pd.eval supports the \& and $|$ bitwise operators. In addition, it supports the use of the literal and and or in Boolean expressions.
    \item \textbf{Object attributes and indices}: pd.eval supports access to object attributes via the obj.attr syntax and indexes via the obj[index] syntax.
    \item \textbf{Other operations}
          Other operations, such as function calls, conditional statements, loops, and other
          more involved constructs are currently \textbf{not} implemented in pd.eval. If you'd like
          to execute these more complicated types of expressions, you can use the
          NumExpr library itself.
\end{itemize}

\section{DataFrame.eval for Column-Wise Operations}
Just as Pandas has a top-level pd.eval function, DataFrame objects have an eval
method that works in similar ways. The benefit of the eval method is that columns
can be referred to by name.

Using pd.eval as in the previous section, we can compute expressions with the three
columns like this:

\begin{pyc}
    df = pd.DataFrame(rng.random(size=(1000, 3)), columns=['A', 'B', 'C'])
    result1 = (df['A'] + df['B']) / (df['C'] - 1)
    result2 = pd.eval('(df.A + df.B) / (df.C - 1)')
    assert np.allclose(result1, result2)
\end{pyc}

The DataFrame.eval method allows much more succinct evaluation of expressions
with the columns:

\begin{pyc}
    result3 = df.eval('(A + B) / (C - 1)')
    assert np.allclose(result1, result3)
\end{pyc}

Notice here that we \textbf{treat column names as variables} within the evaluated expression, and the result is what we would wish.

\subsection*{Assignment in DataFrame.eval}
In addition to the options just discussed, DataFrame.eval also allows assignment to
any column.

In the same way, any existing column can be modified.
\subsection*{Local Variables in DataFrame.eval}
The DataFrame.eval method supports an additional syntax that lets it work with
local Python variables. 需要在变量前面添加@。

The @ character marks a variable name rather than a column name, and lets you
efficiently evaluate expressions involving the two “namespaces”: the namespace of
columns, and the namespace of Python objects. \important{Notice that this @ character is only supported by the DataFrame.eval method, not by the pandas.eval function, because
    the pandas.eval function only has access to the one (Python) namespace.}

\section{The DataFrame.query Method}
The DataFrame has another method based on evaluated strings, called query\marginpar[query]{query}.

\section{Performance: When to Use These Functions}
When considering whether to use eval and query, there are two considerations: computation time and memory use. Memory use is the most predictable aspect. As already
mentioned, every compound expression involving NumPy arrays or Pandas Data
Frames will result in implicit creation of temporary arrays.

If the size of the temporary DataFrames is significant compared to your available system memory (typically several gigabytes), then it's a good idea to use an eval or
query expression. You can check the approximate size of your array in bytes.

On the performance side, eval can be faster even when you are not maxing out your
system memory. The issue is how your temporary objects compare to the size of the
L1 or L2 CPU cache on your system (typically a few megabytes); if they are much bigger, then eval can avoid some potentially slow movement of values between the different memory caches. In practice, I find that the difference in computation time
between the traditional methods and the eval/query method is usually not significant—if anything, the traditional method is faster for smaller arrays! The benefit of
eval/query is mainly in the saved memory, and the sometimes cleaner syntax they
offer.