\chapter{Customizing Ticks\label{}}
Matplotlib's default tick locators and formatters are designed to be generally sufficient
in many common situations, but are in no way optimal for every plot.

Before we go into examples, however, let's talk a bit more about the object hierarchy
of Matplotlib plots. Matplotlib aims to have a Python object representing everything
that appears on the plot: for example, recall that the Figure is the bounding box
within which plot elements appear. Each Matplotlib object can also act as a container
of subobjects: for example, each Figure can contain one or more Axes objects, each
of which in turn contains other objects representing plot contents.

The tickmarks are no exception. Each axes has attributes xaxis and yaxis, which in
turn have attributes that contain all the properties of the lines, ticks, and labels that
make up the axes.
\section{Major and Minor Ticks}
Within each axes, there is the concept of a major tickmark, and a minor tickmark. As
the names imply, major ticks are usually bigger or more pronounced, while minor
ticks are usually smaller. By default, Matplotlib rarely makes use of minor ticks, but
one place you can see them is within logarithmic plots.


These tick properties—locations and labels, that is—can be customized by setting the
\verb|formatter|\marginpar[formatter]{formatter} and \verb|locator|\marginpar[locator]{locator} objects of each axis.



\begin{pyc}
    ax.xaxis.get_major_locator()
    ax.xaxis.get_minor_locator()
    ax.xaxis.get_major_formatter()
    ax.xaxis.get_minor_formatter()
\end{pyc}

\section{Hiding Ticks or Labels}
Perhaps the most common tick/label formatting operation is the act of hiding ticks or
labels. This can be done using \verb|plt.NullLocator| and \verb|plt.NullFormatter|. Having no ticks at
all can be useful in many situations—for example, when you want to show a grid of
images.

\section{Reducing or Increasing the Number of Ticks}
One common problem with the default settings is that smaller subplots can end up
with crowded labels. Particularly for the x-axis ticks, the numbers nearly overlap, making them quite difficult to decipher. One way to adjust this is with \verb|plt.MaxNLocator|, which allows us to
specify the maximum number of ticks that will be displayed.

\section{Fancy Tick Formats}
Matplotlib's default tick formatting can leave a lot to be desired: it works well as a
broad default, but sometimes you'd like to do something different. 比如说需要绘制 $\pi$ 作为刻度,而不是整数或者说是小数表示的 $\pi$。

We can do this by setting a
\verb|MultipleLocator|\marginpar[MultipleLocator]{MultipleLocator}, which locates ticks at a multiple of the number we provide.

we can change the tick formatter. There's no built-in formatter for what we want to do, so
we'll instead use \verb|plt.FuncFormatter|, which accepts a user-defined function giving
fine-grained control over the tick outputs

\section{Summary of Formatters and Locators}
We've seen a couple of the available formatters and locators; I'll conclude this chapter
by listing all of the built-in locator options (\autoref{Matplotlib locator options}) and formatter options
(\autoref{Matplotlib formatter options}).

\begin{table}
    \centering
    \caption{Matplotlib locator options}
    \label{Matplotlib locator options}
    \begin{tabular}{ll}
        \hline
        Locator class    & Description                                             \\
        \hline
        NullLocator      & No ticks                                                \\
        FixedLocator     & Tick locations are fixed                                \\
        IndexLocator     & Locator for index plots (e.g., where x = range(len(y))) \\
        LinearLocator    & Evenly spaced ticks from min to max                     \\
        LogLocator       & Logarithmically spaced ticks from min to max            \\
        MultipleLocator  & Ticks and range are a multiple of base                  \\
        MaxNLocator      & Finds up to a max number of ticks at nice locations     \\
        AutoLocator      & (Default) MaxNLocator with simple defaults              \\
        AutoMinorLocator & Locator for minor ticks                                 \\
        \hline
    \end{tabular}
\end{table}

\begin{table}
    \centering
    \caption{Matplotlib formatter options}
    \label{Matplotlib formatter options}
    \begin{tabular}{ll}
        \hline
        Formatter class    & Description                             \\
        \hline
        NullFormatter      & No labels on the ticks                  \\
        IndexFormatter     & Set the strings from a list of labels   \\
        FixedFormatter     & Set the strings manually for the labels \\
        FuncFormatter      & User-defined function sets the labels   \\
        FormatStrFormatter & Use a format string for each value      \\
        ScalarFormatter    & Default formatter for scalar values     \\
        LogFormatter       & Default formatter for log axes          \\
        \hline
    \end{tabular}
\end{table}