\chapter{Data Indexing and Selection\label{Ch14}}
If you have
used the NumPy patterns, the corresponding patterns in Pandas will feel very familiar, though there are a few quirks to be aware of.
\section{Data Indexing and Selection}
As you saw in the previous chapter, a Series object acts in many ways like a one-
dimensional NumPy array, and in many ways like a standard Python dictionary. If
you keep these two overlapping analogies in mind, it will help you understand the
patterns of data indexing and selection in these arrays.
\subsection*{Series as Dictionary}
Like a dictionary, the Series object provides a mapping from a collection of keys to a
collection of values.

We can also use dictionary-like Python expressions and methods to examine the
keys/indices and values.

Series objects can also be modified with a dictionary-like syntax. Just as you can
extend a dictionary by assigning to a new key, you can extend a Series by assigning
to a new index value.

\subsection*{Series as One-Dimensional Array}
A Series builds on this dictionary-like interface and provides array-style item selec‐
tion via the same basic mechanisms as NumPy arrays—that is, slices, masking, and
fancy indexing.

\begin{tcolorbox}
    Notice that when slicing
    with an explicit index (e.g., \verb|data['a':'c']|), the final index is included in the slice,
    while when slicing with an implicit index (e.g., \verb|data[0:2]|), the final index is excluded
    from the slice.
\end{tcolorbox}

\subsection*{Indexers: loc and iloc}
If your Series has an explicit integer index, an indexing operation such as \verb|data[1]|
will use the explicit indices, while a slicing operation like \verb|data[1:3]| will use the
implicit Python-style indices.

Because of this potential confusion in the case of integer indexes, Pandas provides
some special indexer attributes that explicitly expose certain indexing schemes. These
are not functional methods, but attributes that expose a particular slicing interface to
the data in the Series.

\important{First, the loc\marginpar[loc]{loc} attribute allows indexing and slicing that always references the explicit
    index.}

\important{
    The iloc\marginpar[iloc]{iloc} attribute allows indexing and slicing that always references the implicit
    Python-style index.
}

One guiding principle of Python code is that “explicit is better than implicit.” The
explicit nature of loc and iloc makes them helpful in maintaining clean and readable
code; especially in the case of integer indexes, using them consistently can prevent
subtle bugs due to the mixed indexing/slicing convention.

\section{Data Selection in DataFrames}
Recall that a DataFrame acts in many ways like a two-dimensional or structured array,
and in other ways like a dictionary of Series structures sharing the same index.

\subsection*{DataFrame as Dictionary}
The first analogy we will consider is the DataFrame as a dictionary of related Series
objects.

The individual Series that make up the columns of the DataFrame can be accessed
via dictionary-style indexing of the column name. Equivalently, we can use attribute-style access with column names that are strings.

Though this is a useful shorthand, keep in mind that it does not work for all cases!
For example, if the column names are not strings, or if the column names conflict
with methods of the DataFrame, this attribute-style access is not possible. In particular, you should avoid the temptation to try column assignment via
attributes.

\subsection*{DataFrame as Two-Dimensional Array}
We can examine the raw underlying data array using the values\marginpar[values]{values}
attribute.

When it comes to indexing of a DataFrame object, however, it is clear that the
dictionary-style indexing of columns precludes our ability to simply treat it as a
NumPy array. In particular, passing a single index to an array accesses a row and passing a single “index” to a DataFrame accesses a column.

Any of the familiar NumPy-style data access patterns can be used within these indexers. For example, in the loc indexer we can combine masking and fancy indexing.

\subsection*{Additional Indexing Conventions}
There are a couple of extra indexing conventions that might seem at odds with the
preceding discussion, but nevertheless can be useful in practice.

\begin{itemize}
    \item First, \important{while indexing
              refers to columns, slicing refers to rows}.
    \item Similarly, \important{direct masking operations are interpreted row-wise rather than column-wise}
\end{itemize}
