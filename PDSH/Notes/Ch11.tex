\chapter{Sorting Arrays}
This chapter covers algorithms related to sorting values in
NumPy arrays.

Python has a couple of built-in functions and methods for sorting lists and other
iterable objects. The \verb|sorted|\marginpar[sorted]{sorted} function accepts a list and returns a sorted version of it.

By contrast, the \verb|sort| \marginpar[sort]{sort} method of lists will sort the list in-place.

Python's sorting methods are quite flexible, and can handle any iterable object.
\section{Fast Sorting in NumPy: np.sort and np.argsort}
The \verb|np.sort| function is analogous to Python's built-in sorted function, and will efficiently return a sorted copy of an array.

Similarly to the \verb|sort| method of Python lists, you can also sort an array in-place using
the array \verb|sort| method.

A related function is \verb|argsort|, which instead returns the indices of the sorted elements.

The resulted indices can then be used (via fancy indexing) to construct the sorted array if desired.

\section{Sorting Along Rows or Columns}
A useful feature of NumPy's sorting algorithms is the ability to sort along specific
rows or columns of a multidimensional array using the axis argument.

Keep in mind that this treats each row or column as an independent array, and any
relationships between the row or column values will be lost!

\section{Partial Sorts: Partitioning}
Sometimes we're not interested in sorting the entire array, but simply want to find the
$k$ smallest values in the array. NumPy enables this with the np.partition function.
np.partition takes an array and a number $k$; the result is a new array with the smallest $k$ values to the left of the partition and the remaining values to the right. Within the
two partitions, the elements have arbitrary order.

Finally, just as there is an np.argsort function that computes indices of the sort,
there is an \verb|np.argpartition| function that computes indices of the partition.

