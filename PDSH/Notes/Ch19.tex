\chapter{Combining Datasets: merge and join\label{Ch19}}
One important feature offered by Pandas is its high-performance, in-memory join
and merge operations, which you may be familiar with if you have ever worked with
databases. The main interface for this is the \verb|pd.merge|\marginpar[merge]{merge} function.
\section{Relational Algebra}
The behavior implemented in pd.merge is a subset of what is known as relational
algebra, which is a formal set of rules for manipulating relational data that forms the
conceptual foundation of operations available in most databases. The strength of the relational algebra approach is that it proposes several fundamental operations, which
become the building blocks of more complicated operations on any dataset.

\section{Categories of Joins}
The pd.merge function implements a number of types of joins: \emph{one-to-one}, \emph{many-to-one}, and \emph{many-to-many}. All three types of joins are accessed via an identical call to
the pd.merge interface; the type of join performed depends on the form of the input
data.

\subsection*{One-to-One Joins}
Perhaps the simplest type of merge is the one-to-one join, which is in many ways similar to the column-wise concatenation in \autoref{Ch18}.


The pd.merge function recognizes that each DataFrame has an employee column, and
automatically joins using this column as a key. The result of the merge is a new Data
Frame that combines the information from the two inputs. Notice that the order of
entries in each column is not necessarily maintained: in this case, the order of the
employee column differs between df1 and df2, and the pd.merge function correctly
accounts for this. Additionally, keep in mind that the merge in general discards the
index, except in the special case of merges by index.

\subsection*{Many-to-One Joins}

Many-to-one joins are joins in which one of the two key columns contains duplicate
entries.

\subsection*{Many-to-Many Joins}
If the key column in both the left and right arrays contains duplicates, then the result is a many-to-many merge.

These three types of joins can be used with other Pandas tools to implement a wide
array of functionality.
\section{Specification of the Merge Key}
We've already seen the default behavior of pd.merge: it looks for one or more matching column names between the two inputs, and uses this as the key. However, often
the column names will not match so nicely, and pd.merge provides a variety of
options for handling this.
\subsection*{The on Keyword}
Most simply, you can explicitly specify the name of the key\marginpar[key]{key} column using the on keyword, which takes a column name or a list of column names. This option works only if both the left and right DataFrames have the specified
column name.
\subsection*{The left\_on and right\_on Keywords}

At times you may wish to merge two datasets with different column names; for example, we may have a dataset in which the employee name is labeled as ``name'' rather
than ``employee''. In this case, we can use the \verb|left_on| and \verb|right_on| keywords to
specify the two column names.(在不同的表中,可能字段名不一样,但表示的意思是一样的)

The result has a redundant column that we can drop if desired—for example, by
using the \verb|DataFrame.drop()| method.
\subsection*{The left\_index and right\_index Keywords}
You can use the index as the key for merging by specifying the \verb|left_index| and/or \verb|right_index| flags in pd.merge().

For convenience, Pandas includes the DataFrame.join() method, which performs an
index-based merge without extra keyword.

If you'd like to mix indices and columns, you can combine \verb|left_index| with \verb|right_on|
or \verb|left_on| with \verb|right_index| to get the desired behavior.

All of these options also work with multiple indices and/or multiple columns; the
interface for this behavior is very intuitive. For more information on this, see the
``\href{http://pandas.pydata.org/pandas-docs/stable/user_guide/merging.html}{Merge, Join, and Concatenate}'' section of the Pandas documentation.
\section{Specifying Set Arithmetic for Joins}

We have glossed over one important consideration in
performing a join: the type of set arithmetic used in the join. This comes up when a
value appears in one key column but not the other.


We can specify this explicitly using the how keyword,
which defaults to ``inner''.

Other options for the how keyword are 'outer', 'left', and 'right'. An outer join
returns a join over the union of the input columns and fills in missing values with NAs.

The left join and right join return joins over the left entries and right entries, respectively. All of these options can be applied straightforwardly to any of the preceding join
types.
\section{Overlapping Column Names: The suffixes Keyword}
当 merge 的数据框存在冲突的命名时,考虑使用前缀/后缀。

Because the output would have two conflicting column, the merge function
automatically appends the suffixes \verb|_x| and \verb|_y| to make the output columns unique. If
these defaults are inappropriate, it is possible to specify a custom suffix using the \verb|suffixes|\marginpar[suffixes]{suffixes} keyword。

\section{Example: US States Data}