\chapter{Computation on Arrays: Broadcasting\label{Ch08}}
This chapter discusses \textbf{broadcasting}\marginpar[broadcasting]{broadcasting}: a
set of rules by which NumPy lets you apply binary operations (e.g., addition, subtraction, multiplication, etc.) between arrays of different sizes and shapes.
\section{Introducing Broadcasting}
Recall that for arrays of the same size, binary operations are performed on an
element-by-element basis. Broadcasting allows these types of binary operations to be performed on arrays of dif‐
ferent sizes.

\figures{Visualization of NumPy broadcasting}

\section{Rules of Broadcasting}
Broadcasting in NumPy follows a strict set of rules to determine the interaction
between the two arrays:
\begin{enumerate}
    \item Rule 1: If the two arrays differ in their number of dimensions, the shape of the one with fewer dimensions is padded with ones on its leading (left) side.
    \item Rule 2: If the shape of the two arrays does not match in any dimension, the array with shape equal to 1 in that dimension is stretched to match the other shape.
    \item Rule 3: If in any dimension the sizes disagree and neither is equal to 1, an error is raised.
\end{enumerate}

\subsection{Broadcasting Example 1}
Suppose we want to add a two-dimensional array to a one-dimensional array:
\begin{pyc}
    M = np.ones((2, 3))
    a = np.arange(3)
\end{pyc}

Let's consider an operation on these two arrays, which have the following shapes:
\begin{itemize}
    \item \verb|M.shape| is (2, 3)
    \item \verb|a.shape| is (3,)
\end{itemize}

We see by rule 1 that the array a has fewer dimensions, so we pad it on the left with
ones:
\begin{itemize}
    \item \verb|M.shape| is (2, 3)
    \item \verb|a.shape| is (1, 3)
\end{itemize}

By rule 2, we now see that the first dimension disagrees, so we stretch this dimension
to match:
\begin{itemize}
    \item \verb|M.shape| is (2, 3)
    \item \verb|a.shape| is (2, 3)
\end{itemize}

\subsection{Broadcasting Example 2}
Now let's take a look at an example where both arrays need to be broadcast:

\begin{pyc}
    a = np.arange(3).reshape((3, 1))
    b = np.arange(3)
    a.shape, b.shape
    # ((3, 1), (3,))
\end{pyc}

Again, we'll start by determining the shapes of the arrays:
\begin{itemize}
    \item \verb|a.shape| is (3, 1)
    \item \verb|b.shape| is (3,)
\end{itemize}

Rule 1 says we must pad the shape of b with ones:
\begin{itemize}
    \item \verb|a.shape| is (3, 1)
    \item \verb|b.shape| is (1, 3)
\end{itemize}

And rule 2 tells us that we must upgrade each of these 1s to match the corresponding
size of the other array:
\begin{itemize}
    \item \verb|a.shape| is (3, 3)
    \item \verb|b.shape| is (3, 3)
\end{itemize}

\subsection{Broadcasting Example 3}
Next, let's take a look at an example in which the two arrays are not compatible:

\begin{pyc}
    M = np.ones((3, 2))
    a = np.arange(3)
    M.shape, a.shape
    # ((3, 2), (3,))
\end{pyc}

This is just a slightly different situation than in the first example: the matrix M is
transposed. How does this affect the calculation? The shapes of the arrays are as
follows:

\begin{itemize}
    \item \verb|M.shape| is (3, 2)
    \item \verb|a.shape| is (3,)
\end{itemize}

Again, rule 1 tells us that we must pad the shape of a with ones:

\begin{itemize}
    \item \verb|M.shape| is (3, 2)
    \item \verb|a.shape| is (1, 3)
\end{itemize}

By rule 2, the first dimension of a is then stretched to match that of M:

\begin{itemize}
    \item \verb|M.shape| is (3, 2)
    \item \verb|a.shape| is (3, 3)
\end{itemize}

Now we hit rule 3—the final shapes do not match, so these two arrays are incompatible.

\section{Broadcasting in Practice}