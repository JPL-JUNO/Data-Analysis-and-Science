\chapter{Simple Scatter Plots\label{Ch27}}
Another commonly used plot type is the simple scatter plot, a close cousin of the line
plot. Instead of points being joined by line segments, here the points are represented
individually with a dot, circle, or other shape.

\section{Scatter Plots with plt.plot}

In the previous chapter we looked at using plt.plot/ax.plot to produce line plots. It
turns out that this same function can produce scatter plots as well. The third argument in the function call is a character that represents the type of symbol used for the plotting. Just as you can specify options such as '-' or '--' to control the line style, the marker style has its own set of short string codes.

Most of the possibilities are fairly intuitive, and a number of
the more common ones are demonstrated here (see \autoref{Demonstration of point numbers}).

\figures{Demonstration of point numbers}

For even more possibilities, these character codes can be used together with line and
color codes to plot points along with a line connecting them.

\section{Scatter Plots with plt.scatter}
A second, more powerful method of creating scatter plots is the \verb|plt.scatter|\marginpar[plt.scatter]{plt.scatter} function, which can be used very similarly to the plt.plot function.

\begin{tcolorbox}
    The primary difference of plt.scatter from plt.plot is that it can be used to create
    scatter plots where the properties of each individual point (size, face color, edge color,
    etc.) can be individually controlled or mapped to data.
\end{tcolorbox}

\section{plot Versus scatter: A Note on Efficiency}
\begin{tcolorbox}
    Aside from the different features available in plt.plot and plt.scatter, why might
    you choose to use one over the other? While it doesn’t matter as much for small
    amounts of data, as datasets get larger than a few thousand points, plt.plot can be
    noticeably more efficient than plt.scatter. The reason is that plt.scatter has the
    capability to render a different size and/or color for each point, so the renderer must
    do the extra work of constructing each point individually. With plt.plot, on the
    other hand, the markers for each point are guaranteed to be identical, so the work of
    determining the appearance of the points is done only once for the entire set of data.
    For large datasets, this difference can lead to vastly different performance, and for
    this reason, plt.plot should be preferred over plt.scatter for large datasets.
\end{tcolorbox}

\section{Visualizing Uncertainties}
For any scientific measurement, accurate accounting of uncertainties is nearly as
important, if not more so, as accurate reporting of the number itself.(对任何一种科学测量方法来说,准确地衡量数据误差都是无比重要的事情,甚至比数据
本身还要重要。)
\subsection*{Basic Errorbars}
One standard way to visualize uncertainties is using an errorbar. A basic errorbar can
be created with a single Matplotlib function call, as shown in \autoref{An errorbar example}.

\figures{An errorbar example}

Using these additional options you can easily customize the aesthetics of your errorbar plot. I often find it helpful, especially in crowded plots, to make
the errorbars lighter than the points themselves.

\subsection*{Continuous Errors}
In some situations it is desirable to show errorbars on continuous quantities. Though
Matplotlib does not have a built-in convenience routine for this type of application,
it’s relatively easy to combine primitives like \verb|plt.plot| and \verb|plt.fill_between| for a
useful result.

Here we’ll perform a simple \emph{Gaussian process regression} which is a method of fitting a very flexible nonparametric function to data with a continuous measure of the uncertainty.


\autoref{Representing continuous uncertainty with filled regions}, the resulting figure gives an intuitive view into what the Gaussian process regression
algorithm is doing: in regions near a measured data point, the model is strongly constrained, and this is reflected in the small model uncertainties. In regions far from a
measured data point, the model is not strongly constrained, and the model uncertainties increase.

\figures{Representing continuous uncertainty with filled regions}