\chapter{Hierarchical Indexing\label{Ch17}}
A far more common pattern for handling
higher-dimensional data is to make use of hierarchical indexing (also known as multi-
indexing) to incorporate multiple index levels within a single index.
\section{A Multiply Indexed Series}
Let’s start by considering how we might represent two-dimensional data within a
one-dimensional Series.

Our tuple-based indexing is essentially a
rudimentary multi-index, and the Pandas MultiIndex type gives us the types of oper‐
ations we wish to have. We can create a multi-index from the tuples.

The MultiIndex represents multiple levels of indexing as well as multiple labels for each data point which encode these
levels.

In this multi-index representation, any blank entry indicates the
same value as the line above it.

\subsection*{MultiIndex as Extra Dimension}

The unstack\marginpar[unstack]{unstack} method will quickly convert a multiply
indexed Series into a conventionally indexed DataFrame. Naturally, the stack\marginpar[stack]{stack} method provides the opposite operation.

你可能会纠结于为什么要费时间研究层级索引。其实理由很简单:如果我们可以用含多级
索引的一维 Series 数据表示二维数据,那么我们就可以用 Series 或 DataFrame 表示三维
甚至更高维度的数据。多级索引每增加一级,就表示数据增加一维,利用这一特点就可以
轻松表示任意维度的数据了。

In addition, all the ufuncs and other functionality discussed in \autoref{Ch15} work with
hierarchical indices as well.
\section{Methods of MultiIndex Creation}
The most straightforward way to construct a multiply indexed Series or DataFrame
is to simply pass a list of two or more index arrays to the constructor.(我个人觉得这种两层索引或者少量层索引还有合适,多层索引的话 List 嵌套根本看不出来)

Similarly, if you pass a dictionary with appropriate tuples as keys, Pandas will automatically recognize this and use a MultiIndex by default.

\subsection*{Explicit MultiIndex Constructors}
For more flexibility in how the index is constructed, you can instead use the constructor methods available in the \verb|pd.MultiIndex| class.

You can construct a MultiIndex from a simple list of arrays giving the index values
within each level, or you can construct it from a list of tuples giving the multiple index values of each point.

ou can even construct it from a Cartesian product of single indices.

Similarly, you can construct a MultiIndex directly using its internal encoding by
passing levels (a list of lists containing available index values for each level) and
codes (a list of lists that reference these labels).

Any of these objects can be passed as the index argument when creating a Series or
DataFrame, or be passed to the reindex method of an existing Series or DataFrame.

\subsection*{MultiIndex Level Names}

Sometimes it is convenient to name the levels of the MultiIndex. This can be accomplished by passing the names argument to any of the previously discussed MultiIndex
constructors, or by setting the names\marginpar[names]{names} attribute of the index after the fact.

\subsection*{MultiIndex for Columns}
In a DataFrame, the rows and columns are completely symmetric, and just as the rows
can have multiple levels of indices, the columns can have multiple levels as well.
\section{Indexing and Slicing a MultiIndex}
\subsection*{Multiply Indexed Series}
We can access single elements by indexing with multiple terms.

The MultiIndex also supports partial indexing, or indexing just one of the levels in
the index. The result is another Series, with the lower-level indices maintained.

Partial slicing is available as well, as long as the MultiIndex is sorted.(需要索引是排序的) With sorted indices, partial indexing can be performed on lower levels by passing an
empty slice in the first index.

Other types of indexing and selection (discussed in \autoref{Ch14}) work as well.

\subsection*{Multiply Indexed DataFrames}

Remember that columns are primary in a DataFrame, and the syntax used for multiply indexed Series applies to the columns.

Also, as with the single-index case, we can use the loc, iloc, and ix indexers introduced in \autoref{Ch14}.

These indexers provide an array-like view of the underlying two-dimensional data,
but each individual index in loc or iloc can be passed a tuple of multiple indices.

Working with slices within these index tuples is not especially convenient; trying to
create a slice within a tuple will lead to a syntax error

You could get around this by building the desired slice explicitly using Python’s built-
in slice function, but a better way in this context is to use an IndexSlice\marginpar[IndexSlice]{IndexSlice} object,
which Pandas provides for precisely this situation.

\section{Rearranging Multi-Indexes}
One of the keys to working with multiply indexed data is knowing how to effectively
transform the data. There are a number of operations that will preserve all the information in the dataset, but rearrange it for the purposes of various computations.
\subsection{Sorted and Unsorted Indices}
\important{Many of the
    MultiIndex slicing operations will fail if the index is not sorted.}

We’ll start by creating some simple multiply indexed data where the indices are not
lexographically sorted(不是按照字典排序的).


For various reasons, partial slices and other similar operations require the levels in the MultiIndex to be in sorted (i.e., lexographical) order.
Pandas provides a number of convenience routines to perform this type of sorting,
such as the \verb|sort_index| and \verb|sortlevel| methods of the DataFrame.

\subsection*{Stacking and Unstacking Indices}
As we saw briefly before, it is possible to convert a dataset from a stacked multi-index
to a simple two-dimensional representation, optionally specifying the level to use.

The opposite of unstack\marginpar[unstack]{unstack} is stack\marginpar[stack]{stack}, which here can be used to recover the original
series.

\subsection*{Index Setting and Resetting}
Another way to rearrange hierarchical data is to turn the index labels into columns;
this can be accomplished with the \verb|reset_index|\marginpar[reset\_index]{reset\_index} method. Calling this on the population dictionary will result in a DataFrame with state and year columns holding the
information that was formerly in the index. For clarity, we can optionally specify the
name of the data for the column representation.

A common pattern is to build a MultiIndex from the column values. This can be
done with the \verb|set_index|\marginpar[set\_index]{set\_index} method of the DataFrame, which returns a multiply indexed
DataFrame