\chapter{Customizing Plot Legends\label{Ch29}}
Plot legends give meaning to a visualization, assigning meaning to the various plot
elements.

We can specify the location and turn on the frame. We can use the ncol command to specify the number of columns in the legend. And we can use a rounded box (fancybox) or add a shadow, change the transparency
(alpha value) of the frame, or change the padding around the text(see \autoref{A fancybox plot legend}).

\figures{A fancybox plot legend}
\section{Choosing Elements for the Legend}
As we have already seen, by default the legend includes all labeled elements from the
plot. If this is not what is desired, we can fine-tune which elements and labels appear
in the legend by using the objects returned by plot commands. plt.plot is able to
create multiple lines at once, and returns a list of created line instances. Passing any of
these to plt.legend will tell it which to identify, along with the labels we'd like to
specify.

Notice that the legend ignores all elements without a label attribute set.


\section{Legend for Size of Points}
Sometimes the legend defaults are not sufficient for the given visualization. For example, perhaps you're using the size of points to mark certain features of the data, and
want to create a legend reflecting this. (可视化气泡图,需要将图里的大小也同步调整)

The legend will always reference some object that is on the plot, so if we'd like to display a particular shape we need to plot it. In this case, the objects we want (gray circles) are not on the plot, so we fake them by plotting empty lists. Recall that the
legend only lists plot elements that have a label specified.

\section{Multiple Legends}

Sometimes when designing a plot you'd like to add multiple legends to the same axes.
Unfortunately, Matplotlib does not make this easy: via the standard legend interface,
it is only possible to create a single legend for the entire plot. If you try to create a
second legend using plt.legend or ax.legend, it will simply override the first one.
We can work around this by creating a new legend artist from scratch (\verb|Artist|\marginpar[Artist]{Artist} is the
base class Matplotlib uses for visual attributes), and then using the lower-level
\verb|ax.add_artist| method to manually add the second artist to the plot.

\figures{A split plot legend}

\begin{pyc}
    fig, ax = plt.subplots()
    lines = []
    styles = ['-', '--', '-.', ':']
    x = np.linspace(0, 10, 1000)
    for i in range(4):
    lines += ax.plot(x, np.sin(x - i * np.pi / 2),
    styles[i], color='black')
    ax.legend(lines[:2], ['line A', 'line B'], loc='upper right')

    ax.axis('equal')

    from matplotlib.legend import Legend
    leg = Legend(ax, lines[2:], ['line C', 'line D'], loc='lower right')
    ax.add_artist(leg)
\end{pyc}