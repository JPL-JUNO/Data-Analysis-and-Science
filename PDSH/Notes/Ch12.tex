\chapter{Structured Data: NumPy's Structured Arrays\label{Ch12}}
\section{Exploring Structured Array Creation}
We can use dictionary method to create structured array, which key is names and formats.

For clarity, numerical types can be specified using Python types or NumPy dtypes
instead.

A compound type can also be specified as a list of tuples. If the names of the types do not matter to you, you can specify the types alone in a
comma-separated string.

The shortened string format codes may not be immediately intuitive, but they are
built on simple principles. The first (optional) character $<$ or $>$, means “little endian”(低字节序)
or “big endian,” respectively, and specifies the ordering convention for significant
bits. The next character specifies the type of data: characters, bytes, ints, floating
points, and so on (see \autoref{NumPy data types}). The last character or characters represent the size
of the object in bytes.

\begin{table}
    \centering
    \caption{NumPy data types}
    \label{NumPy data types}
    \begin{tabular}{lll}
        \hline
        Character & Description            & Example                                 \\
        \hline
        'b'       & Byte                   & \verb|np.dtype('b')|                    \\
        'i'       & Signed integer         & \verb|np.dtype('i4') == np.int32|       \\
        'u'       & Unsigned integer       & \verb|np.dtype('u1') == np.uint8|       \\
        'f'       & Floating point         & \verb|np.dtype('f8') == np.int64|       \\
        'c'       & Complex floating point & \verb|np.dtype('c16') == np.complex128| \\
        'S', 'a'  & String                 & \verb|np.dtype('S5')|                   \\
        'U'       & Unicode string         & \verb|np.dtype('U') == np.str_|         \\
        'V'       & Raw data (void)        & \verb|np.dtype('V') == np.void|         \\
        \hline
    \end{tabular}
\end{table}

\section{More Advanced Compound Types}

It is possible to define even more advanced compound types. For example, you can
create a type where each element contains an array or matrix of values.

为什么我们宁愿用这种方法存
储数据,也不用简单的多维数组,或者 Python 字典呢?原因是 NumPy 的 dtype 直接映射
到 C 结构的定义,因此包含数组内容的缓存可以直接在 C 程序中使用。如果你想写一个
Python 接口与一个遗留的 C 语言或 Fortran 库交互,从而操作结构化数据,你将会发现结
构化数组非常有用!
\section{Record Arrays: Structured Arrays with a Twist}
NumPy also provides record arrays (instances of the np.recarray class), which are
almost identical to the structured arrays just described, but with one additional feature: fields can be accessed as attributes rather than as dictionary keys.

If we view our data as a record array instead, we can access this with slightly fewer
keystrokes. The downside is that for record arrays, there is some extra overhead involved in
accessing the fields, even when using the same syntax.

Whether the more convenient notation is worth the (slight) overhead will depend on
your own application.

\section{On to Pandas}
This chapter on structured and record arrays is purposely located at the end of this
part of the book, because it leads so well into the next package we will cover: Pandas.
Structured arrays can come in handy in certain situations, like when you’re using
NumPy arrays to map onto binary data formats in C, Fortran, or another language.