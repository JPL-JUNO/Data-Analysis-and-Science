\chapter{Working with Time Series\label{Ch23}}

Pandas was originally developed in the context of financial modeling, so as you might
expect, it contains an extensive set of tools for working with dates, times, and time-
indexed data. Date and time data comes in a few flavors, which we will discuss here:

\begin{itemize}
    \item Timestamps(时间戳)
          Particular moments in time (e.g., July 4, 2021 at 7:00 a.m.).
    \item Time intervals(时间间隔) and periods(周期)
          A length of time between a particular beginning and end point; for example, the
          month of June 2021. Periods usually reference a special case of time intervals in
          which each interval is of uniform length and does not overlap (e.g., 24-hour-long
          periods comprising days).
    \item Time deltas(时间增量或者时间差) or durations(持续时间)
          An exact length of time (e.g., a duration of 22.56 seconds).
\end{itemize}

\section{Dates and Times in Python}
The Python world has a number of available representations of dates, times, deltas,
and time spans. While the time series tools provided by Pandas tend to be the most
useful for data science applications, it is helpful to see their relationship to other tools
used in Python.
\subsection*{Native Python Dates and Times: datetime and dateutil}
Python's basic objects for working with dates and times reside in the built-in
datetime module. Along with the third-party dateutil module, you can use this to
quickly perform a host of useful functionalities on dates and times.

The power of datetime and dateutil lies in their flexibility and easy syntax: you can
use these objects and their built-in methods to easily perform nearly any operation
you might be interested in. Where they break down is when you wish to work with
large arrays of dates and times: just as lists of Python numerical variables are suboptimal compared to NumPy-style typed numerical arrays, lists of Python datetime
objects are suboptimal compared to typed arrays of encoded dates.
\subsection*{Typed Arrays of Times: NumPy's datetime64}
NumPy's datetime64 dtype encodes dates as 64-bit integers, and thus allows arrays of
dates to be represented compactly and operated on in an efficient manner. The
datetime64 requires a specific input format. Once we have dates in this form, we can quickly do vectorized operations on it.

One detail of the datetime64 and related timedelta64 objects is that they are built on
a \emph{fundamental time unit}. Because the datetime64 object is limited to 64-bit precision,
the range of encodable times is $2^{64}$ times this fundamental unit. In other words,
datetime64 imposes a trade-off between time resolution and maximum time span.

For example, if you want a time resolution of 1 nanosecond, you only have enough
information to encode a range of $2^{64}$ nanoseconds, or just under 600 years. NumPy
will infer the desired unit from the input.

You can force any desired fundamental unit using one of many format codes. \autoref{tab23-1}, drawn from the NumPy datetime64 documentation, lists the available
format codes along with the relative and absolute time spans that they can encode.

\begin{table}
    \centering
    \caption{Description of date and time codes}
    \label{tab23-1}
    \begin{tabular}{llll}
        \hline
        Code & Meaning     & Time span (relative) & Time span (absolute)   \\
        \hline
        Y    & Year        & $\pm$ 9.2e18 years   & [9.2e18 BC, 9.2e18 AD] \\
        M    & Month       & $\pm$ 7.6e17 years   & [7.6e17 BC, 7.6e17 AD] \\
        W    & Week        & $\pm$ 1.7e17 years   & [1.7e17 BC, 1.7e17 AD] \\
        D    & Day         & $\pm$ 2.5e16 years   & [2.5e16 BC, 2.5e16 AD] \\
        h    & Hour        & $\pm$ 1.0e15 years   & [1.0e15 BC, 1.0e15 AD] \\
        m    & Minute      & $\pm$ 1.7e13 years   & [1.7e13 BC, 1.7e13 AD] \\
        s    & Second      & $\pm$ 2.9e12 years   & [ 2.9e9 BC, 2.9e9 AD]  \\
        ms   & Millisecond & $\pm$ 2.9e9 years    & [ 2.9e6 BC, 2.9e6 AD]  \\
        us   & Microsecond & $\pm$ 2.9e6 years    & [290301 BC, 294241 AD] \\
        ns   & Nanosecond  & $\pm$ 292 years      & [ 1678 AD, 2262 AD]    \\
        ps   & Picosecond  & $\pm$ 106 days       & [ 1969 AD, 1970 AD]    \\
        fs   & Femtosecond & $\pm$ 2.6 hours      & [ 1969 AD, 1970 AD]    \\
        as   & Attosecond  & $\pm$ 9.2 seconds    & [ 1969 AD, 1970 AD]    \\
        \hline
    \end{tabular}
\end{table}

For the types of data we see in the real world, a useful default is datetime64[ns], as it
can encode a useful range of modern dates with a suitably fine precision.

Finally, note that while the datetime64 data type addresses some of the deficiencies of
the built-in Python datetime type, it lacks many of the convenient methods and
functions provided by datetime and especially dateutil.

\subsubsection*{Dates and Times in Pandas: The Best of Both Worlds}
Pandas builds upon all the tools just discussed to provide a Timestamp object, which
combines the ease of use of datetime and dateutil with the efficient storage and
vectorized interface of numpy.datetime64. From a group of these Timestamp objects,
Pandas can construct a DatetimeIndex that can be used to index data in a Series or
DataFrame.

\section{Pandas Time Series: Indexing by Time}
The Pandas time series tools really become useful when you begin to index data by
timestamps.

\section{Pandas Time Series Data Structures}
This section will introduce the fundamental Pandas data structures for working with
time series data:
\begin{itemize}
    \item For timestamps, Pandas provides the Timestamp type. As mentioned before, this
          is essentially a replacement for Python's native datetime, but it's based on the
          more efficient numpy.datetime64 data type. The associated Index structure is
          DatetimeIndex.
    \item For time periods, Pandas provides the Period type. This encodes a fixed-
          frequency interval based on numpy.datetime64. The associated index structure is
          PeriodIndex.
    \item For time deltas or durations, Pandas provides the Timedelta type. Timedelta is a
          more efficient replacement for Python's native datetime.timedelta type, and is
          based on numpy.timedelta64. The associated index structure is TimedeltaIndex.
\end{itemize}

The most fundamental of these date/time objects are the Timestamp and Datetime
Index objects. While these class objects can be invoked directly, it is more common to
use the \verb|pd.to_datetime| function, which can parse a wide variety of formats. Passing
a single date to \verb|pd.to_datetime| yields a Timestamp; passing a series of dates by
default yields a DatetimeIndex.

Any DatetimeIndex can be converted to a PeriodIndex with the \verb|to_period| function,
with the addition of a frequency code.

A TimedeltaIndex is created when a date is subtracted from another.

\section{Regular Sequences: pd.date\_range}
To make creation of regular date sequences more convenient, Pandas offers a few
functions for this purpose: \verb|pd.date_range| for timestamps, \verb|pd.period_range| for
periods, and \verb|pd.timedelta_range| for time deltas.

\verb|pd.date_range| accepts a start date, an end date, and an optional
frequency code to create a regular sequence of dates. Alternatively, the date range can be specified not with a start and end point, but with
a start point and a number of periods. The spacing can be modified by altering the freq argument, which defaults to D.


To create regular sequences of Period or Timedelta values, the similar
\verb|pd.period_range| and \verb|pd.timedelta_range| functions are useful.

\section{Frequencies and Offsets}
Fundamental to these Pandas time series tools is the concept of a frequency or date
offset. \autoref{tab23-2} summarizes the main codes available.
\begin{table}
    \centering
    \caption{Listing of Pandas frequency codes}
    \label{tab23-2}
    \begin{tabular}{llll}
        \hline
        Code & Description  & Code & Description          \\
        \hline
        D    & Calendar day & B    & Business day         \\
        W    & Weekly       &      &                      \\
        M    & Month end    & BM   & Business month end   \\
        Q    & Quarter end  & BQ   & Business quarter end \\
        A    & Year end     & BA   & Business year end    \\
        H    & Hours        & BH   & Business hours       \\
        T    & Minutes      &      &                      \\
        S    & Seconds      &      &                      \\
        L    & Milliseconds &      &                      \\
        U    & Microseconds &      &                      \\
        N    & Nanoseconds  &      &                      \\
        \hline
    \end{tabular}
\end{table}

The monthly, quarterly, and annual frequencies are all marked at the end of the specified period. Adding an S suffix to any of these causes them to instead be marked at
the beginning (see \autoref{tab23-3}).


Additionally, you can change the month used to mark any quarterly or annual code
by adding a three-letter month code as a suffix(比如说美股的财年是任意的):
\begin{itemize}
    \item Q-JAN, BQ-FEB, QS-MAR, BQS-APR, etc.
    \item A-JAN, BA-FEB, AS-MAR, BAS-APR, etc.
\end{itemize}

In the same way, the split point of the weekly frequency can be modified by adding a
three-letter weekday code: W-SUN, W-MON, W-TUE, W-WED, etc.

All of these short codes refer to specific instances of Pandas time series offsets, which
can be found in the pd.tseries.offsets module.

\begin{table}
    \centering
    \caption{Listing of start-indexed frequency codes}
    \label{tab23-3}
    \begin{tabular}{llll}
        \hline
        Code & Description   & Code & Description            \\
        \hline
        MS   & Month start   & BMS  & Business month start   \\
        QS   & Quarter start & BQS  & Business quarter start \\
        AS   & Year start    & BAS  & Business year start    \\
        \hline
    \end{tabular}
\end{table}

\section{Resampling, Shifting, and Windowing}
The ability to use dates and times as indices to intuitively organize and access data is
an important aspect of the Pandas time series tools. The benefits of indexed data in
general (automatic alignment during operations, intuitive data slicing and access,
etc.) still apply, and Pandas provides several additional time series–specific
operations.
\subsection*{Resampling and Converting Frequencies}
One common need when dealing with time series data is resampling at a higher or
lower frequency. This can be done using the resample\marginpar[resample]{resample} method, or the much simpler
asfreq\marginpar[asfreq]{asfreq} method. \important{The primary difference between the two is that resample is fundamentally a data aggregation, while asfreq is fundamentally a data selection.}

Let's compare what the two return when we downsample(下采样) the S\&P 500 closing price
data.\autoref{Resampling of SP500 closing price} shows
the result.
\figures{Resampling of SP500 closing price}
Notice the difference(\autoref{Resampling of SP500 closing price}): at each point, \tips{resample reports the average of the previous year,
    while asfreq reports the value at the end of the year.}

For upsampling, resample and asfreq are largely equivalent, though resample has
many more options available. In this case, the default for both methods is to leave the
upsampled points empty; that is, filled with NA values. asfreq accepts a method argument to specify how values are
imputed. Here, we will resample the business day data at a daily frequency (i.e.,
including weekends); \autoref{Comparison between forward-fill and back-fill interpolation} shows the result. Because the S\&P 500 data only exists for business days, the top panel has gaps representing NA values. The bottom panel shows the differences between two strategies for
filling the gaps: forward filling and backward filling.

\figures{Comparison between forward-fill and back-fill interpolation}
\subsection*{Time Shifts}
Another common time series–specific operation is shifting of data in time. For this,
Pandas provides the shift\marginpar[shift]{shift} method, which can be used to shift data by a given number of entries. With time series data sampled at a regular frequency, this can give us a
way to explore trends over time.(比如说要计算资产收益率,一年期的)

\subsection*{Rolling Windows}
Calculating rolling statistics is a third type of time series–specific operation implemented by Pandas. This can be accomplished via the rolling attribute of Series and
DataFrame objects, which returns a view similar to what we saw with the groupby
operation (see \autoref{Ch20}). This rolling view makes available a number of aggregation operations by default.