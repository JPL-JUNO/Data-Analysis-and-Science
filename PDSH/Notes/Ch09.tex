\chapter{Comparisons, Masks, and Boolean Logic}
Masking comes up when you want to extract, modify, count, or
otherwise manipulate values in an array based on some criterion.
In NumPy, Boolean masking is often the most efficient
way to accomplish these types of tasks.
\section{Example: Counting Rainy Days}

We saw in \autoref{Ch06} that NumPy's ufuncs can be used in place of loops to do
fast element-wise arithmetic operations on arrays; in the same way, we can use other
ufuncs to do element-wise comparisons over arrays
\section{Comparison Operators as Ufuncs}

NumPy implements comparison operators such as $<$ (less than) and $>$
(greater than) as element-wise ufuncs. The result of these comparison operators is
always an array with a Boolean data type. All six of the standard comparison operations are available.

As in the case of arithmetic operators, the comparison operators are implemented as
ufuncs in NumPy; for example, when you write \verb|x < 3|, internally NumPy uses
\verb|np.less(x, 3)|. A summary of the comparison operators and their equivalent ufuncs
is shown here:

\begin{table}[H]
    \centering
    \begin{tabular}{llll}
        \hline
        Operator & Equivalent ufunc  & Operator & Equivalent ufunc        \\
        \hline
        $==$     & \verb|np.equal|   & $!=$     & \verb|np.not_equal|     \\
        $<$      & \verb|np.less|    & $<=$     & \verb|np.less_equal|    \\
        $>$      & \verb|np.greater| & $>=$     & \verb|np.greater_equal| \\
        \hline
    \end{tabular}
\end{table}
\section{Working with Boolean Arrays}
\subsection*{Counting Entries}
To count the number of True entries in a Boolean array, \verb|np.count_nonzero| is useful. Another way to get at this
information is to use np.sum; in this case, False is interpreted as 0, and True is interpreted as 1.

The benefit of np.sum is that, like with other NumPy aggregation functions, this summation can be done along rows or columns as well.

If we're interested in quickly checking whether any or all the values are True, we can
use (you guessed it) \verb|np.any| or \verb|np.all|. \verb|np.all| and \verb|np.any| can be used along particular axes as well.

\subsection*{Boolean Operators}
The following table summarizes the bitwise Boolean operators and their equivalent
ufuncs:

\begin{table}[H]
    \centering
    \begin{tabular}{llll}
        \hline
        Operator & Equivalent ufunc      & Operator & Equivalent ufunc      \\
        \hline
        $\&$     & \verb|np.bitwise_and| & $|$      & \verb|np.bitwise_or|  \\
        \^{}     & \verb|np.bitwise_xor| & \~{}     & \verb|np.bitwise_not| \\
        \hline
    \end{tabular}
\end{table}

\section{Boolean Arrays as Masks}
In the preceding section we looked at aggregates computed directly on Boolean
arrays. A more powerful pattern is to use Boolean arrays as masks, to select particular
subsets of the data themselves.

To select values from array, we can simply index on this Boolean array;
this is known as a \textbf{masking}\marginpar[masking]{masking} operation. What is \notes{returned is a one-dimensional array} filled with all the values that meet this
condition; in other words, all the values in positions at which the mask array is True.

\section{Using the Keywords and/or Versus the Operators $\&/|$}
One common point of confusion is the difference between the keywords and and or
on the one hand, and the operators $\&$ and $|$ on the other. When would you use one versus the other?

The difference is this: and and or operate on the object as a whole, while $\&$ and $|$
operate on the elements within the object.

When you use and or or, it is equivalent to asking Python to treat the object as a single Boolean entity. In Python, all nonzero integers will evaluate as True.

When you use $\&$ and $|$ on integers, the expression operates on the bitwise representation of the element, applying the and or the or to the individual bits making up the
number.

When you have an array of Boolean values in NumPy, this can be thought of as a
string of bits where \verb|1 = True| and \verb|0 = False|, and \& and $|$ will operate.

But if you use or on these arrays it will try to evaluate the truth or falsehood of the
entire array object, which is not a well-defined value.

Similarly, when evaluating a Boolean expression on a given array, you should use $|$ or
\& rather than or or and.

Trying to evaluate the truth or falsehood of the entire array will give the same
\verb|ValueError|.

So, remember this: and and or perform a single Boolean evaluation on an entire
object, while \& and $|$ perform multiple Boolean evaluations on the content (the individual bits or bytes) of an object. For Boolean NumPy arrays, the latter is nearly
always the desired operation.