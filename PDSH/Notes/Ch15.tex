\chapter{Operating on Data in Pandas\label{Ch15}}
One of the strengths of NumPy is that it allows us to perform quick element-wise
operations, both with basic arithmetic (addition, subtraction, multiplication, etc.) and
with more complicated operations (trigonometric functions, exponential and logarithmic functions, etc.). Pandas inherits much of this functionality from NumPy, and
the ufuncs introduced in \autoref{Ch06} are key to this.

Pandas includes a couple of useful twists, however: for unary operations like negation
and trigonometric functions, these ufuncs will preserve index and column labels in the
output(保留index和column), and for binary operations such as addition and multiplication, Pandas will
automatically align indices(自动对齐索引进行运算) when passing the objects to the ufunc. This means that
keeping the context of data and combining data from different sources—both potentially error-prone tasks with raw NumPy arrays—become essentially foolproof with
Pandas.

\section{Ufuncs: Index Preservation}
Because Pandas is designed to work with NumPy, any NumPy ufunc will work on
Pandas Series and DataFrame objects.

\section{Ufuncs: Index Alignment}
For binary operations on two Series or DataFrame objects, Pandas will align indices
in the process of performing the operation. This is very convenient when working
with incomplete data.

\subsection*{Index Alignment in Series}

The resulting array contains the union of indices of the two input arrays, which could
be determined directly from these indices.

Any item for which one or the other does not have an entry is marked with NaN, or
“Not a Number,” which is how Pandas marks missing data. This index matching is implemented this way for any of
Python’s built-in arithmetic expressions; any missing values are marked by NaN.

If using NaN values is not the desired behavior, the fill value can be modified using
appropriate object methods in place of the operators. For example, calling \verb|A.add(B)|
is equivalent to calling A + B, but allows optional explicit specification of the fill value
for any elements in A or B that might be missing.(交集执行运算,差集使用fill\_value执行运算)

\subsection*{Index Alignment in DataFrames}

A similar type of alignment takes place for \notes{both columns and indices} when performing operations on DataFrame objects.

Notice that indices are aligned correctly irrespective of their order in the two objects,
and indices in the result are sorted.

\autoref{Mapping between Python operators and Pandas methods} lists Python operators and their equivalent Pandas object methods.

\begin{table}
    \centering
    \caption{Mapping between Python operators and Pandas methods}
    \label{Mapping between Python operators and Pandas methods}
    \begin{tabular}{ll}
        \hline
        Python operator & Pandas method(s)     \\
        \hline
        \verb|+|        & add                  \\
        \verb|-|        & sub, subtract        \\
        \verb|*|        & mul, multiply        \\
        \verb|/|        & truediv, div, divide \\
        \verb|//|       & floordiv             \\
        \verb|%|        & mod                  \\
        \verb|**|       & pow                  \\
        \hline
    \end{tabular}
\end{table}
\section{Ufuncs: Operations Between DataFrames and Series}
When performing operations between a DataFrame and a Series, the index and column alignment is similarly maintained, and the result is similar to operations
between a two-dimensional and one-dimensional NumPy array.

In Pandas, the convention similarly operates row-wise by default.

If you would instead like to operate column-wise, you can use the object methods
mentioned earlier, while specifying the axis keyword.

This preservation and alignment of indices and columns means that operations on
data in Pandas will always maintain the data context, which prevents the common
errors that might arise when working with heterogeneous and/or misaligned data in
raw NumPy arrays.
