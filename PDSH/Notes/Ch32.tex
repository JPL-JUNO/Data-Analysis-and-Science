\chapter{Text and Annotation\label{Ch32}}
Creating a good visualization involves guiding the reader so that the figure tells a
story. In some cases, this story can be told in an entirely visual manner, without the
need for added text, but in others, small textual cues and labels are necessary. Perhaps
the most basic types of annotations you will use are axes labels and titles, but the
options go beyond this.

When we're visualizing data, it is often useful to annotate certain features of
the plot to draw the reader's attention. This can be done manually with the \verb|plt.text|\marginpar[plt.text]{plt.text}/
\verb|ax.text|\marginpar[ax.text]{ax.text} functions, which will place text at a particular $x$/$y$ value.

The ax.text method takes an $x$ position, a $y$ position, a string, and then optional keywords specifying the color, size, style, alignment, and other properties of the text.
Here we used \verb|ha='right'| and \verb|ha='center'|, where \verb|ha| is short for \emph{horizontal alignment}.

\section{Transforms and Text Position}
In the previous example, we anchored our text annotations to data locations. Sometimes it's preferable to anchor the text to a fixed position on the axes or figure, independent of the data. In Matplotlib, this is done by modifying the transform.

Matplotlib makes use of a few different coordinate systems: a data point at
$(x, y) = (1, 1)$ corresponds to a certain location on the axes or figure, which in turn
corresponds to a particular pixel on the screen. Mathematically, transforming
between such coordinate systems is relatively straightforward, and Matplotlib has a
well-developed set of tools that it uses internally to perform these transforms.

A typical user rarely needs to worry about the details of the transforms, but it is helpful knowledge to have when considering the placement of text on a figure. There are
three predefined transforms that can be useful in this situation:
\begin{itemize}
    \item \textbf{ax.transData}: Transform associated with data coordinates
    \item \textbf{ax.transAxes}: Transform associated with the axes (in units of axes dimensions)
    \item \textbf{fig.transFigure}: Transform associated with the figure (in units of figure dimensions)
\end{itemize}

\figures{Comparing Matplotlib coordinate systems}

The transData coordinates give the usual data coordinates associated with the x- and
y-axis labels. The transAxes coordinates give the location from the bottom-left corner of the axes (the white box), as a fraction of the total axes size. The transFig
ure coordinates are similar, but specify the position from the bottom-left corner of
the figure (the gray box) as a fraction of the total figure size.

\important{Notice now that if we change the axes limits, it is only the transData coordinates that
    will be affected, while the others remain stationary. }

\section{Arrows and Annotation}
Along with tickmarks and text, another useful annotation mark is the simple arrow.

I'd suggest using the \verb|plt.annotate|\marginpar[plt.annotate]{plt.annotate} function, which creates some text and an arrow and allows the arrows to be very flexibly
specified.

More discussion and examples of available arrow and annotation styles can be found
in the Matplotlib \href{https://matplotlib.org/stable/tutorials/text/annotations.html}{Annotations tutorial}.

\figures{Annotated average birth rates by day}