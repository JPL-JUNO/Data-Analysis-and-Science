\chapter{Aggregations: min, max, and Everything in Between\label{Ch07}}
\section{Summing the Values in an Array}

Python内置的sum函数和Numpy提供的sum函数执行的结果都是一样的,但是\verb|np.sum|执行的效率更快。

Be careful, though: the sum function and the np.sum function are not identical, which
can sometimes lead to confusion! In particular, their optional arguments have different meanings (sum(x, 1) initializes the sum at 1, while np.sum(x, 1) sums along
axis 1), and np.sum is aware of multiple array dimensions.

\section{Minimum and Maximum}
Similarly, Python has built-in min and max functions, used to find the minimum value
and maximum value of any given array. NumPy's corresponding functions have similar syntax, and again operate much more
quickly.

For \verb|min|, \verb|max|, \verb|sum|, and several other NumPy aggregates, a shorter syntax is to use
methods of the array object itself.

\subsection{Multidimensional Aggregates}
One common type of aggregation operation is an aggregate along a row or column.

NumPy aggregations will apply across all elements of a multidimensional array.

Aggregation functions take an additional argument specifying the \verb|axis|\marginpar[axis]{axis} along which
the aggregate is computed. The axis keyword specifies the dimension of the array that will be collapsed,
rather than the dimension that will be returned. So, specifying axis=0 means that axis
0 will be collapsed: for two-dimensional arrays, values within each column will be
aggregated.

\subsection{Other Aggregation Functions}
NumPy provides several other aggregation functions with a similar API, and additionally most have a NaN-safe counterpart that computes the result while ignoring
missing values, which are marked by the special IEEE floating-point NaN value.

\autoref{Aggregation functions available in NumPy} provides a list of useful aggregation functions available in NumPy.
\begin{table}
    \centering
    \caption{Aggregation functions available in NumPy}
    \label{Aggregation functions available in NumPy}
    \begin{tabular}{lll}
        \hline
        Function name & NaN-safe version & Description                               \\
        \hline
        np.sum        & np.nansum        & Compute sum of elements                   \\
        np.prod       & np.nanprod       & Compute product of elements               \\
        np.mean       & np.nanmean       & Compute mean of elements                  \\
        np.std        & np.nanstd        & Compute standard deviation                \\
        np.var        & np.nanvar        & Compute variance                          \\
        np.min        & np.nanmin        & Find minimum value                        \\
        np.max        & np.nanmax        & Find maximum value                        \\
        np.argmin     & np.nanargmin     & Find index of minimum value               \\
        np.argmax     & np.nanargmax     & Find index of maximum value               \\
        np.median     & np.nanmedian     & Compute median of elements                \\
        np.percentile & np.nanpercentile & Compute rank-based statistics of elements \\
        np.any        & N/A              & Evaluate whether any elements are true    \\
        np.all        & N/A              & Evaluate whether all elements are true    \\
        \hline
    \end{tabular}
\end{table}
\section{Example: What Is the Average Height of US Presidents?}