\chapter{An Introduction to Streamlit\label{ch01}}
\section{}
st.pyplot() is a function that lets us use all the power of the popular matplotlib library and push our matplotlib graph to Streamlit. Once we create a figure in matplotlib, we can explicitly tell Streamlit to write that to our app with the st.pyplot() function.
\section{Finishing touches – adding text to Streamlit}
Other than st.write(), we also can utilize other built-in functions that format our text for us, such as st.title(), st.header(), st.markdown(), and st.subheader(). Using these five functions helps to format text in our Streamlit apps easily and keeps sizing consistent for bigger apps.

More specifically, st.title() will place a large block of text in our app, st.header() uses a slightly smaller font than st.title(), and st.subheader() uses an even smaller one. Other than those three, st.markdown() will allow anyone already familiar with Markdown to use the popular markup language in our Streamlit apps.