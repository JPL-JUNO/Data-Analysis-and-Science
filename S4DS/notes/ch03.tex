\chapter{Data Visualization\label{ch03}}
\section{}
\section{Streamlit visualization use cases}
Some Streamlit users are relatively experienced Python developers with well-tested workflows in visualization libraries of their choice. For these users, the best path forward is the one we’ve taken so far, which is to create graphs in our library of choice (Seaborn, Matplotlib, Bokeh, and so on) and then use the appropriate Streamlit function to write this to the app.

Other Streamlit users will have less experience in Pythonic graphing, and especially for these users, Streamlit offers a few built-in functions.

\subsection{Streamlit’s built-in graphing functions}
There are four built-in functions for graphing – \verb|st.line_chart()|, \verb|st.bar_chart()|, \verb|st.area_chart()|, and \verb|st.map()|.
\subsection{Bokeh}
We can call Bokeh graphs using the same format as Plotly. First, we create the Bokeh graph, and then we use the \verb|st.bokeh_chart()| function to write the app to Streamlit. In Bokeh, we have to first instantiate a Bokeh figure object, and then change aspects of that figure before we can plot it out. The important lesson here is that if we change an aspect of the Bokeh figure object after we call the \verb|st.bokeh_chart()| function, we will not change the graph shown on the Streamlit app.