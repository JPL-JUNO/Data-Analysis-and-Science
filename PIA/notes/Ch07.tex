\chapter{MultiIndex DataFrames\label{Ch07}}
\section{The MultiIndex object}
To summarize, a MultiIndex is a storage container in which each label holds multiple values. A level consists of the values at the same position across the labels.

\section{MultiIndex DataFrames}

\section{Sorting a MultiIndex}
Pandas can find a value in an ordered collection much quicker than in a jumbled one. A good analogous example is searching for a word in a dictionary. It's easier to locate a word when words are in alphabetical order rather than a random sequence. Thus, \textbf{it's optimal to sort an index before selecting any rows and columns from a DataFrame}.

When we invoke the method on a MultiIndex DataFrame, pandas sorts all levels in ascending order and proceeds from the outside in.
\section{Selecting with a MultiIndex}
\subsection{Extracting one or more columns}
If we pass a single value in square brackets, pandas will look for it in the outermost level of the columns' MultiIndex.

To specify values across multiple levels in the column's MultiIndex, we can pass them inside a tuple.

To extract multiple DataFrame columns, we need to pass the square brackets a list of tuples. Each tuple should specify the level values for one column. The order of tuples within the list sets the order of columns in the resulting DataFrame.
\section{}
\section{}