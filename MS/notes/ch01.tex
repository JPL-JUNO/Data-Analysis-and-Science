\chapter{Your First Shiny App\label{ch01}}
\section{添加 UI 控件}
\begin{itemize}
    \item fluidPage() 是一个布局函数,用于设置页面的基本视觉结构。
    \item selectInput() 是一个输入控件,允许用户通过提供值与应用程序交互。
    \item verbatimTextOutput() 和 tableOutput() 是输出控件,告诉 Shiny 将渲染输出放在哪里。verbatimTextOutput()显 示代码并 tableOutput()显 示表格。
\end{itemize}
\section{使用反应式表达式减少重复}
您可以通过包装一段代码并将 \verb|reactive({...})| 其分配给变量来创建反应式表达式,并且可以通过像函数一样调用它来使用反应式表达式。但是,虽然看起来您正在调用函数,但响应式表达式有一个重要的区别:它仅在第一次调用时运行,然后缓存其结果,直到需要更新为止。