\chapter{响应式基础\label{ch03}}
\section{server 函数}
\subsection{input}
参数 input 是一个类似列表的对象,其中包含从浏览器发送的所有输入数据,根据输入 ID 命名。与普通的列表不同,input 对象是只读的。如果你尝试在服务函数内的修改输入,你将收到错误。发生此错误是因为 input 反映了浏览器中发生的情况,而浏览器是 Shiny 的“单一事实来源”。如果你可以修改 R 中的值,则可能会导致不一致,即输入滑块在浏览器中表示一件事,而 input\$count 在 R 中表示不同的内容。这将使编程变得具有挑战性!稍后,在 \autoref{ch08} 中,你将学习如何使用诸如 updateNumericInput() 修改浏览器中的值之类的功能,然后 input\$count 进行相应的更新。

关于 input 更重要的一件事是:它对谁可以阅读是有选择性的。\textbf{要读取 input,必须处于由 renderText() 或 reactive() 函数创建的反应式上下文中。}
\subsection{输出}
output 与 input 非常相似:它也是一个根据输出 ID 命名的类似列表的对象。主要区别在于使用它来发送输出而不是接收输入。你总是要把 output 对象与 render 函数结合使用。

渲染函数做了两件事:
\begin{itemize}
    \item 它设置了一个特殊的反应上下文,可以自动跟踪输出使用的输入。

    \item 它将 R 代码的输出转换为适合在网页上显示的 HTML。
\end{itemize}
与 input 一样 ,output 对如何使用它很挑剔。
\section{响应式编程}
Shiny 的重要思想:你不需要告诉输出何时更新,因为 Shiny 会自动为你计算出来。
\subsection{命令式编程 imperative programming 与声明式 declarative programming 编程}
命令和  recipes 之间的区别是两种重要编程风格之间的主要区别之一:

\begin{itemize}
    \item 在命令式编程中,你发出特定命令,它会立即执行。这是你在分析脚本中习惯的编程风格:命令 R 加载数据、转换数据、可视化数据,并将结果保存到磁盘。
    \item 在声明式编程中,你表达更高级别的目标或描述重要的约束,并依靠其他人来决定如何和/或何时将其转化为行动。这是你在 Shiny 中使用的编程风格。
\end{itemize}
命令式代码是 assertive;声明式代码是 passive-aggressive。
\subsection{反应图}
\figures{fig3-2}{反应图显示了输入和输出的连接方式}
反应图是了解应用程序工作原理的强大工具。随着你的应用程序变得越来越复杂,制作反应图的快速高级草图通常很有用,以提醒你所有部分如何组合在一起。在本书中,我们将向你展示反应图,以帮助你理解示例的工作原理,稍后在 \autoref{ch14} 中,你将学习如何使用 \href{https://cran.r-project.org/web/packages/reactlog/index.html}{reactlog} 来为你绘制图表。
\subsection{响应表达式}
你将在反应图中看到一个更重要的组件:反应表达式。反应式表达式接受输入并产生输出,因此它们具有结合输入和输出特征的形状。希望这些形状能帮助你记住组件如何组合在一起。
\subsection{执行顺序}
重要的是要理解代码运行的顺序完全由反应图决定。这与大多数 R 代码不同,大多数 R 代码的执行顺序由行的顺序决定。
\section{响应表达式}
反应式表达式具有输入和输出的风格:
\begin{itemize}
    \item 与输入一样,您可以在输出中使用反应式表达式的结果。
    \item 与输出一样,反应式表达式依赖于输入并自动知道何时需要更新。
\end{itemize}
这种二元性意味着我们需要一些新的词汇:我将使用生产者(producer)来指代反应式输入和表达式,使用消费者(consumer)来指代反应式表达式和输出。
\figures{fig3-5}{输入和表达式是反应式生产者;表达式和输出是反应式消费者}