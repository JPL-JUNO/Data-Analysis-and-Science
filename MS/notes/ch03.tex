\chapter{响应式基础\label{ch03}}
\section{server 函数}
\subsection{input}
参数 input 是一个类似列表的对象,其中包含从浏览器发送的所有输入数据,根据输入 ID 命名。与普通的列表不同,input 对象是只读的。如果你尝试在服务函数内的修改输入,你将收到错误。发生此错误是因为 input 反映了浏览器中发生的情况,而浏览器是 Shiny 的“单一事实来源”。如果你可以修改 R 中的值,则可能会导致不一致,即输入滑块在浏览器中表示一件事,而 input\$count 在 R 中表示不同的内容。这将使编程变得具有挑战性!稍后,在 \autoref{ch08} 中,你将学习如何使用诸如 updateNumericInput() 修改浏览器中的值之类的功能,然后 input\$count 进行相应的更新。

关于 input 更重要的一件事是:它对谁可以阅读是有选择性的。\textbf{要读取 input,必须处于由 renderText() 或 reactive() 函数创建的响应式上下文中。}
\subsection{输出}
output 与 input 非常相似:它也是一个根据输出 ID 命名的类似列表的对象。主要区别在于使用它来发送输出而不是接收输入。你总是要把 output 对象与 render 函数结合使用。

渲染函数做了两件事:
\begin{itemize}
    \item 它设置了一个特殊的响应上下文,可以自动跟踪输出使用的输入。

    \item 它将 R 代码的输出转换为适合在网页上显示的 HTML。
\end{itemize}
与 input 一样 ,output 对如何使用它很挑剔。
\section{响应式编程}
Shiny 的重要思想:你不需要告诉输出何时更新,因为 Shiny 会自动为你计算出来。
\subsection{命令式编程 imperative programming 与声明式 declarative programming 编程}
命令和  recipes 之间的区别是两种重要编程风格之间的主要区别之一:

\begin{itemize}
    \item 在命令式编程中,你发出特定命令,它会立即执行。这是你在分析脚本中习惯的编程风格:命令 R 加载数据、转换数据、可视化数据,并将结果保存到磁盘。
    \item 在声明式编程中,你表达更高级别的目标或描述重要的约束,并依靠其他人来决定如何和/或何时将其转化为行动。这是你在 Shiny 中使用的编程风格。
\end{itemize}
命令式代码是 assertive;声明式代码是 passive-aggressive。
\subsection{响应图}
\figures{fig3-2}{响应图显示了输入和输出的连接方式}
响应图是了解应用程序工作原理的强大工具。随着你的应用程序变得越来越复杂,制作响应图的快速高级草图通常很有用,以提醒你所有部分如何组合在一起。在本书中,我们将向你展示响应图,以帮助你理解示例的工作原理,稍后在 \autoref{ch14} 中,你将学习如何使用 \href{https://cran.r-project.org/web/packages/reactlog/index.html}{reactlog} 来为你绘制图表。
\subsection{响应表达式}
你将在响应图中看到一个更重要的组件:响应表达式。响应式表达式接受输入并产生输出,因此它们具有结合输入和输出特征的形状。希望这些形状能帮助你记住组件如何组合在一起。
\subsection{执行顺序}
重要的是要理解代码运行的顺序完全由响应图决定。这与大多数 R 代码不同,大多数 R 代码的执行顺序由行的顺序决定。
\section{响应表达式}
响应式表达式具有输入和输出的风格:
\begin{itemize}
    \item 与输入一样,你可以在输出中使用响应式表达式的结果。
    \item 与输出一样,响应式表达式依赖于输入并自动知道何时需要更新。
\end{itemize}
这种二元性意味着我们需要一些新的词汇:我将使用生产者(producer)来指代响应式输入和表达式,使用消费者(consumer)来指代响应式表达式和输出。
\figures{fig3-5}{输入和表达式是响应式生产者;表达式和输出是响应式消费者}
\subsection{简化图形}
你可能熟悉编程的“三规则”:每当你将某些内容复制并粘贴三次时,你应该弄清楚如何减少重复(通常通过编写函数)。这很重要,因为它减少了代码中的重复量,这使得代码更容易理解,并且随着需求的变化更容易更新。

然而,在 Shiny 中,我认为你应该考虑一规则:每当你复制并粘贴某些内容时,你应该考虑将重复的代码提取到响应式表达式中。该规则对于 Shiny 来说更为严格,因为响应式表达式不仅使人们更容易理解代码,还提高了 Shiny 有效重新运行代码的能力。
\section{控制评估时间}
\subsection{定时失效}
想象一下,你想通过不断地重新,以便你看到动画而不是静态图。我们可以通过一个新功能来提高更新频率:reactiveTimer()。reactiveTimer() 是一个响应式表达式,依赖于隐藏输入:当前时间。当你希望响应式表达式比其他方式更频繁地使其自身无效时,可以使用 reactiveTimer()。
\subsection{在点击时执行}
假设在一个定时器中,每 0.5 秒执行一次代码,但是代码的执行时间需要 1 秒,那么 Shiny 需要做的事情越来越多。如果用户快速点击更改某个参数同样会直到 Shiny 需要执行的事情越来越多,尤其是在涉及昂贵计算时。

如果你的应用程序中出现这种情况,你可能希望要求用户通过单击按钮来选择执行昂贵的计算。这是 actionButton() 一个很好的用例。

我们需要一个新工具 eventReactive():一种使用输入值而不对其产生响应性依赖的方法,它有两个参数:第一个参数指定要依赖的内容,第二个参数指定要计算的内容。

\autoref{fig3-15} 体现了这种思想。
\figures{fig3-15}{根据需要,lambda1、lambda2、n 不再对 x1、 x2 具有响应性依赖:更改它们的值将不会触发计算。将箭头保留为非常浅的灰色只是为了提醒你 x1 和 x2 继续使用这些值,但不再对它们产生响应性依赖。}
\section{观察员}
有些操作不会在页面中展示,但是需要记录,比如说调式消息、发送数据等等,这应该使用输入和输出的 render,而是需要使用观察员 observers。

observeEvent() 它为你提供了一个重要的调试工具。

observeEvent() 与 eventReactive() 非常相似。它有两个重要的参数:eventExpr 和 handlerExpr。第一个参数是要依赖的输入或表达式;第二个参数是将运行的代码。

observeEvent() 和 eventReactive() 之间有两个重要的区别:
\begin{itemize}
    \item 你没有将 observeEvent() 的结果分配给变量
    \item 你无法从其他响应性消费者那里引用它
\end{itemize}
\figures{fig3-16}{}

观察员和产出密切相关。您可以将输出视为具有特殊的副作用:更新用户浏览器中的 HTML。为了强调这种接近性,我们将在响应图中以相同的方式绘制它们。这会产生 \autoref{fig3-16} 所示的响应图。