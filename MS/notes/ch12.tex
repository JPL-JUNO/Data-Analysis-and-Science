\chapter{Tidy Evaluation\label{ch12}}
\section{动机}
\begin{description}
    \item[env-variable]
        An environment variable is a “programming” variable that you create with \verb|<-|. \verb|input$var| is an env-variable.
    \item[data-variable] A data frame variable is a “statistical” variable that lives inside a data frame. carat is a data-variable.
\end{description}
有了这些新术语,我们可以使间接问题变得更加清晰:我们有一个数据变量(carat) 存储在环境变量(\verb|input$var|) 中,并且我们需要某种方法来告诉 dplyr。有两种略有不同的方法可以执行此操作,具体取决于你正在使用的函数是“数据屏蔽”函数还是“整齐选择”函数。

\section{数据屏蔽}
数据屏蔽函数允许你使用“当前”数据框中的变量,而无需任何额外的语法。它在许多 dplyr 函数中使用,例如 arrange()、filter()、group\_by()、mutate() 和 summarise(),以及 ggplot2 中的 aes()。数据屏蔽很有用,因为它允许你使用数据变量而无需任何附加语法。

在数据屏蔽函数内部,你可以使用 .data 或者 .env 如果你想明确你正在谈论的是数据变量还是环境变量。

\verb|diamonds %>% filter(.data$carat > .env$min)|

\section{整齐选择}
除了数据屏蔽之外,整洁评估还有另一个重要部分:整洁选择。Tidy-selection 提供了一种按位置、名称或类型选择列的简洁方法。它用在 dplyr::select() 和d plyr::across() 以及 tidyr 中的许多函数中,例如 pivot\_longer(), pivot\_wider(), separate(), extract(), 和 unite()。
\subsection{间接}
要间接引用变量,请使用 any\_of() 或则 all\_of():两者都期望包含数据变量名称的字符向量环境变量。唯一的区别是,如果你提供输入中不存在的变量名,会发生什么:all\_of() 会抛出错误,而 any\_of() 会默默地忽略它。

\subsection{Tidy-Selection and Data-Masking}
当你使用使用 tidy-selection 的函数时,使用多个变量非常简单:你只需将变量名称的字符向量传递到 any\_of() 或 all\_of() 即可。如果我们也能在数据屏蔽函数中做到这一点,不是很好吗?这就是 dplyr 1.0.0 中添加 across() 函数的想法。它允许你在数据屏蔽函数中使用整洁选择。

across() 通常与一个或两个参数一起使用。第一个参数选择变量,在 group\_by() 或 distinct() 等函数中很有用。

第二个参数是应用于每个选定列的函数(或函数列表)。这使得它非常适合 mutate() 与 summarise() 当你希望以某种方式转换每个变量的情况。
\section{parse() and eval()}
这是一种很诱人的方法,因为它只需要学习很少的新想法。但它有一些主要缺点:因为您将字符串粘贴在一起,所以很容易意外创建无效代码,或者可能被滥用的代码来执行您不想要的操作。如果它是一个只有您使用的闪亮应用程序,那么这并不是非常重要,但这不是一个好习惯 - 否则很容易在您更广泛共享的应用程序中意外地创建安全漏洞。我们将在\nameref{ch22}中回顾这个想法。

(如果这是你能找到解决问题的唯一方法,你不应该感到难过,但是当你有更多的心理空间时,我建议花一些时间弄清楚如何在不进行字符串操作的情况下做到这一点。这将帮助您成为一名更好的 R 程序员。)