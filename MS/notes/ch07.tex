\chapter{图形\label{ch07}}
\section{交互性}
plotOutput() 最酷的事情之一是,它不仅可以作为显示绘图的输出,还可以作为响应指针事件的输入。这允许你创建交互式图形,用户可以直接与绘图上的数据进行交互。交互式图形是一种强大的工具,具有广泛的应用范围。
\subsection{基础知识}
绘图可以响应四种不同的鼠标23事件:click、dblclick(双击)、hover(当鼠标在同一位置停留一小会儿时)和brush(矩形选择工具)。要将这些事件转换为闪亮的输入,你可以向相应的参数提供一个字符串 plotOutput(),例如 \verb|plotOutput("plot", click = "plot_click")|。这将创建一个 \verb|input$plot_click| 可用于处理绘图上的鼠标单击的 。

\subsection{点击}
点事件返回一个相对丰富的列表,其中包含大量信息。最重要的组成部分是 x 和 y,它们给出了数据坐标中事件的位置。但我不会谈论这种数据结构,因为你只在相对罕见的情况下需要。相反,你将使用 nearPoints() 帮助程序,它返回一个数据框,其中包含与点击接近的行(一个观测值),处理一堆繁琐的细节。

这里我们给出 nearPoints() 四个参数:绘图下的数据框、输入事件以及轴上变量的名称。如果你使用 ggplot2,则只需提供前两个参数,因为 xvar 和 yvar 可以从绘图数据结构自动估算。

你可能想知道到底 nearPoints() 返回了什么。这是一个使用的好地方 browser()。

另一种使用方法 nearPoints() 是 \verb|allRows = TRUE| 与 \verb|addDist = TRUE| 和 一起使用。这将返回带有两个新列的原始数据框:
\begin{itemize}
    \item \verb|dist_| 给出行和事件之间的距离(以像素为单位)。
    \item \verb|selected_| 表示它是否靠近单击事件(即,是否是在 \verb|allRows = FALSE| 时返回的行)。
\end{itemize}
\subsection{其他点事件}
同样的方法同样适用于 click、dblclick 和 hover:只需更改参数的名称即可。clickOpts() 如果需要,你可以通过提供 dblclickOpts() 或 hoverOpts() 来代替给出输入 ID 的字符串,从而获得对事件的额外控制。

你可以在一张图上使用多种交互类型。只需确保向用户解释他们可以做什么:使用鼠标事件与应用程序交互的一个缺点是它们不能立即被发现。

\subsection{画笔}
在绘图上选择点的另一种方法是使用画笔(brushing),即由四个边定义的矩形选择。click在 Shiny中,一旦你掌握了 click 和 nearPoints() 画笔的使用就很简单:你只需切换到brush 参数和 brushedPoints() 助手即可。

brushOpts() 用于控制颜色 (fill 和 stroke),或 direction = ``x" 或 ``y" 将 brush 限制为单一维度(例如,对于刷牙时间序列很有用)。
\subsection{修改绘图}
当你在与之交互的同一个绘图中显示变化时,交互性的真正魅力就显现出来了。

reactiveVal() 与 reactive() 类似。 你可以通过调用其初始值来创建响应式值 reactiveVal(),并以与响应式相同的方式检索该值。最大的区别在于,你还可以更新响应性值,并且引用它的所有响应性使用者都将重新计算。响应式值使用特殊的语法进行更新 - 你可以像函数一样调用它,第一个参数是新值。

\subsection{交互限制}
在我们继续之前,了解交互式图中的基本数据流以了解它们的局限性非常重要。基本流程是这样的:
\begin{enumerate}
    \item JavaScript 捕获鼠标事件。
    \item Shiny 将鼠标事件数据发送回 R,告诉应用程序输入现在已过时。
    \item 所有下游响应性消费者都被重新计算。
    \item plotOutput() 生成一个新的 PNG 并将其发送到浏览器。
\end{enumerate}
对于本地应用程序,瓶颈往往是绘制绘图所需的时间。根据情 plot 的复杂程度,这可能需要几分之一秒的时间。但对于托管应用程序,您还必须考虑将事件从浏览器传输到 R,然后将渲染的绘图从 R 传输回浏览器所需的时间。
\section{动态高度和宽度}
本章的其余部分不如交互式图形那么令人兴奋,但包含一些需要在某个地方介绍的重要材料。

首先,可以使绘图大小具有响应性,因此宽度和高度会根据用户操作而变化。为此,请为零参数函数 renderPlot() 的参数 width 和 height 提供值 - 现在必须在服务器而不是 UI 中定义这些函数,因为它们可以更改。这些函数应该没有参数并返回所需的像素大小。它们在响应性环境中进行评估,以便您可以动态调整绘图的大小。

\section{图像}
如果您想显示现有图像(而不是绘图),则可以使用 renderImage()。

renderImage()需要返回一个列表。唯一关键的参数是src图像文件的本地路径。您还可以额外提供:

\begin{itemize}
    \item contentType,定义图像的 MIME 类型。如果未提供,Shiny 会根据文件扩展名进行猜测,因此仅当您的图像没有扩展名时才需要提供此文件
    \item 图像的 width 和 height(如果已知)
    \item 任何其他参数(例如class或 alt)将作为属性添加到 HTML \verb|<img>|  中的标记中
\end{itemize}

您还必须提供 deleteFile 值。不幸的是,renderImage() 最初设计用于处理临时文件,因此它在渲染图像后会自动删除图像。这显然是非常危险的,因此在 Shiny 1.5.0 中行为发生了变化。现在,shiny 不再删除图像,而是强制您明确选择您想要的行为。