\chapter{反应式构建块\label{ch15}}
\section{响应值}
有两种类型的响应值:
\begin{itemize}
    \item 由 reactiveVal() 创建的单个响应值。
    \item 由 reactiveValues() 创建的响应值列表。
\end{itemize}

大多数 R 对象都具有 \verb|copy_on_modify| 语义,这意味着如果您将相同的值分配给两个名称,则一旦您修改一个名称,连接就会中断。对于反应性值来说,情况并非如此——它们总是保留对同一值的引用,以便修改任何副本都会修改所有值。
\section{隔离代码}
为了结束本章,我将讨论两个重要的工具,用于精确控制反应图失效的方式和时间。在本节中,我将讨论 isolate(),该工具为 observeEvent() 和 eventReactive() 提供支持,并且可以让您避免在不需要时创建反应性依赖项。在下一节中,您将了解 invalidateLater(),它允许您按计划生成反应性失效。
\subsection{isolate()}
观察者通常与反应值结合在一起,以便跟踪状态随时间的变化。
\subsection{observeEvent() 和 eventReactive()}
observeEvent() 和 eventReactive() 具有允许您控制其操作细节的附加参数:
\begin{enumerate}
    \item 默认情况下,这两个函数都会忽略产生的任何事件NULL(或者在操作按钮的特殊情况下,忽略 0)。使用 ignoreNULL = FALSE 来处理 NULL 值。
    \item 默认情况下,这两个函数在您创建它们时都会运行一次。用 ignoreInit = TRUE 跳过此运行。
    \item 仅对 observeEvent(),您可以设置 once = TRUE 来仅处理程序一次。
\end{enumerate}
\section{定时失效}
isolate() 减少反应图失效的时间。本节的主题 invalidateLater() 做了相反的事情:它允许您在没有数据更改时使反应图无效。
\subsection{轮询}
一个有用的 invalidateLater() 应用是将 Shiny 连接到 R 外部正在更改的数据。