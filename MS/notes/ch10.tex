\chapter{动态 UI\label{ch10}}
创建动态用户界面有以下三种关键技术:
\begin{itemize}
    \item 使用 update 函数族修改输入控件的参数。
    \item 用于 tabsetPanel() 有条件地显示和隐藏部分用户界面。
    \item 使用 uiOutput() 和 renderUI() 通过代码生成用户界面的选定部分。
\end{itemize}

这三个工具使您能够通过修改输入和输出来响应用户。我将演示一些更有用的方法来应用它们,但最终您只会受到您的创造力的限制。同时,这些工具可能会使您的应用程序更加难以推理,因此请谨慎部署它们,并始终努力使用最简单的技术来解决您的问题。
\section{更新输入}
每个输入控件都与一个\textbf{更新函数}配对,该函数允许您在创建控件后对其进行修改。更新函数看起来与其他 Shiny 函数略有不同:它们都将输入的名称(作为字符串)作为 inputId 参数。其余参数对应于输入构造函数的参数,可以在创建后进行修改。

\subsection{分层选择框}
一个特别重要的应用是通过逐步过滤,可以更轻松地从一长串可能的选项中进行选择。这通常是“分层选择框”的问题。更新功能的一个更复杂但特别有用的应用是允许跨多个类别进行交互式钻取。