\chapter{上传和下载\label{ch09}}
与用户之间传输文件是应用程序的常见功能。您可以使用它上传数据进行分析,或将结果下载为数据集或报告。
\section{上传}
\subsection{用户界面}
支持文件上传所需的 UI 很简单:只需添加 fileInput() 到您的 UI 中即可。与大多数其他 UI 组件一样,只有两个必需参数: id 和 label。
\subsection{服务器}
服务器上的处理fileInput()比其他输入稍微复杂一些。大多数输入返回简单向量,但fileInput()返回包含四列的数据框:
\begin{itemize}
    \item name:用户计算机上的原始文件名。
    \item size:文件大小,以字节为单位。默认情况下,用户只能上传最大 5 MB 的文件。您可以通过在启动 Shiny 之前设置 shiny.maxRequestSize 选项来增加此限制。例如,允许最多 10 MB 运行 \verb|options(shiny.maxRequestSize = 10 * 1024^2)|。
    \item type:文件的“MIME 类型” 。这是文件类型的正式规范,通常源自扩展名,并且在 Shiny 应用程序中很少需要。
    \item datapath:数据在服务器上上传的路径。将此路径视为临时路径:如果用户上传更多文件,则该文件可能会被删除。数据始终保存到临时目录并指定临时名称。
\end{itemize}

\subsection{上传数据}
如果用户上传数据集,则需要注意两个细节:
\begin{enumerate}
    \item 在页面加载时 \verb|input$upload| 已初始化为 NULL ,因此您需要 \verb|req(input$upload)| 确保代码等待第一个文件上传。
    \item 参数 accept 允许您限制可能的输入。最简单的方法是提供文件扩展名的字符向量,例如 accept = ".csv"。 但该accept 参数只是对浏览器的建议,并不总是强制执行,因此最好自己验证它。在 R 中获取文件扩展名的最简单方法是 \verb|tools::file_ext()| ,只需注意它会删除扩展名中的前缀.。
\end{enumerate}
\section{下载}
\subsection{基础知识}
同样,用户界面很简单:使用 downloadButton(id) 或 downloadLink(id) 为用户提供单击以下载文件的内容。

与其他输出不同,downloadButton() 它不与渲染函数配对。相反,您使用 downloadHandler()。

downloadHandler()有两个参数,两个都是函数:

\begin{itemize}
    \item filename 应该是一个不带参数的函数,返回文件名(作为字符串)。该函数的作用是创建将在下载对话框中向用户显示的名称。
    \item content 应该是一个带有一个参数的函数,file 该参数是保存文件的路径。该函数的作用是将文件保存在 Shiny 知道的地方,以便它可以将其发送给用户。
\end{itemize}

这是一个不寻常的界面,但它允许 Shiny 控制文件的保存位置(以便可以将其放置在安全位置),同时您仍然可以控制该文件的内容。
\subsection{下载报告}
除了下载数据之外,您可能还希望应用程序的用户下载一份报告,该报告总结了 Shiny 应用程序中交互式探索的结果。这是相当大量的工作,因为您还需要以不同的格式显示相同的信息,但这对于高风险的应用程序非常有用。

生成此类报告的一种有效方法是使用\href{https://bookdown.org/yihui/rmarkdown/parameterized-reports.html}{参数化的 RMarkdown} 文档。参数化的 RMarkdown 文件在 YAML 元数据中有一个 params 字段。

还有一些其他技巧值得了解:
\begin{itemize}
    \item RMarkdown 在当前工作目录中工作,这在许多部署场景中都会失败(例如在 shinyapps.io 上)。您可以通过在应用程序启动时将报告复制到临时目录(即在服务器功能之外)来解决此问题
    \item 默认情况下,RMarkdown 将在当前进程中渲染报表,这意味着它将继承 Shiny 应用程序的许多设置(如加载的包、选项等)。为了提高鲁棒性,我建议使用 callr 包在单独的 R 会话中运行 render()
\end{itemize}

shinymeta 包解决了一个相关问题:有时您需要能够将 Shiny 应用程序的当前状态转换为可以在将来重新运行的可重现报告。