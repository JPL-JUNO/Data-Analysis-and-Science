\chapter{基础 UI\label{ch02}}
\section{输出}
请注意,有两个渲染函数的行为略有不同:
\begin{itemize}
    \item renderText() 将结果组合成一个字符串,并且通常与 textOutput()
    \item renderPrint() 打印结果,就像您在 R 控制台中一样,并且通常与 verbatimTextOutput()
\end{itemize}
\subsection{表格}
有两种用于在表中显示数据框的选项:

\begin{itemize}
    \item tableOutput() 与 renderTable() 渲染一个静态数据表,一次性显示所有数据。

    \item dataTableOutput() 与 renderDataTable() 呈现一个动态表,显示固定数量的行以及用于更改哪些行可见的控件。
\end{itemize}

tableOutput() 对于小型、固定的 summary(例如模型系数)最有用;如果您想向用户公开完整的数据框,则 dataTableOutput() 最合适。
\subsection{绘图}
您可以使用 plotOutput() 和 renderPlot() 显示任何类型的 R 图形(base、ggplot2 或其他)。
\subsection{下载}
您可以让用户使用 downloadButton() 或 downloadLink() 来下载文件。