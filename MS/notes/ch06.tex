\chapter{布局、主题、HTML\label{ch06}}
\section{单页布局}
布局函数提供应用程序的高级视觉结构。布局是由函数调用的层次结构创建的,其中 R 中的层次结构与生成的 HTML 中的层次结构相匹配。这有助于你理解布局代码。
\subsection{页面功能}
最重要但最无趣的布局函数是 fluidPage(),它看起来是一个非常无聊的应用程序,但幕后有很多工作,因为 fluidPage() 设置了 Shiny 所需的所有 HTML、CSS 和 JavaScript。

除了 之外 fluidPage(),Shiny 还提供了一些其他页面函数,可以在更特殊的情况下派上用场:fixedPage() 和 fillPage()。 fixedPage() 工作原理类似 fluidPage(),但有一个固定的最大宽度,这可以防止你的应用程序在更大的屏幕上变得不合理的宽度。 fillPage() 填充浏览器的整个高度,如果你想制作占据整个屏幕的绘图,则非常有用。你可以在他们的文档中找到详细信息。
\subsection{带侧边栏的页面}
要制作更复杂的布局,你需要在 fluidPage(). 例如,要制作一个左侧输入、右侧输出的两列布局,你可以使用 sidebarLayout()(以及它的朋友 titlePanel()、sidebarPanel() 和 mainPanel())。

\figures{fig6-2}{带有侧边栏的基本应用程序的结构}

\figures{fig6-3}{常见的应用程序设计是将控件放在侧边栏中并在主面板中显示结果}
\subsection{多行}
在底层,sidebarLayout() 它构建在灵活的多行布局之上,你可以直接使用它来创建视觉上更复杂的应用程序。像往常一样,你从 fluidPage() 开始。然后,你可以使用 fluidRow() 来创建行,并使用 column() 来创建列。

每行由 12 列组成,column() 第一个参数给出要占用的列数。12 列布局为你提供了极大的灵活性,因为你可以轻松创建 2 列、3 列或 4 列布局,或使用窄列来创建间隔。
\figures{fig6-4}{简单多行应用程序的底层结构}
\section{多页面布局}
随着你的应用程序变得越来越复杂,可能无法将所有内容都放在一个页面上。在本节中,你将学习 tabPanel() 创建多个页面错觉的各种用法。这是一种错觉,因为你仍然拥有一个带有单个底层 HTML 文件的应用程序,但它现在被分成了几部分,并且一次只能看到一个部分。

多页面应用程序与模块配合得特别好,你将在 \autoref{ch18} 中了解这些模块。模块允许你以与划分用户界面相同的方式划分服务器功能,创建仅通过明确定义的连接进行交互的独立组件。
\subsection{选项卡集}
将页面分成多个部分的简单方法是使用 tabsetPanel() 以及与它近似的 tabPanel()。