\chapter{布局、主题、HTML\label{ch06}}
\section{单页布局}
布局函数提供应用程序的高级视觉结构。布局是由函数调用的层次结构创建的,其中 R 中的层次结构与生成的 HTML 中的层次结构相匹配。这有助于你理解布局代码。
\subsection{页面功能}
最重要但最无趣的布局函数是 fluidPage(),它看起来是一个非常无聊的应用程序,但幕后有很多工作,因为 fluidPage() 设置了 Shiny 所需的所有 HTML、CSS 和 JavaScript。

除了 之外 fluidPage(),Shiny 还提供了一些其他页面函数,可以在更特殊的情况下派上用场:fixedPage() 和 fillPage()。 fixedPage() 工作原理类似 fluidPage(),但有一个固定的最大宽度,这可以防止你的应用程序在更大的屏幕上变得不合理的宽度。 fillPage() 填充浏览器的整个高度,如果你想制作占据整个屏幕的绘图,则非常有用。你可以在他们的文档中找到详细信息。
\subsection{带侧边栏的页面}
要制作更复杂的布局,你需要在 fluidPage(). 例如,要制作一个左侧输入、右侧输出的两列布局,你可以使用 sidebarLayout()(以及它的朋友 titlePanel()、sidebarPanel() 和 mainPanel())。

\figures{fig6-2}{带有侧边栏的基本应用程序的结构}

\figures{fig6-3}{常见的应用程序设计是将控件放在侧边栏中并在主面板中显示结果}
\subsection{多行}
在底层,sidebarLayout() 它构建在灵活的多行布局之上,你可以直接使用它来创建视觉上更复杂的应用程序。像往常一样,你从 fluidPage() 开始。然后,你可以使用 fluidRow() 来创建行,并使用 column() 来创建列。

每行由 12 列组成,column() 第一个参数给出要占用的列数。12 列布局为你提供了极大的灵活性,因为你可以轻松创建 2 列、3 列或 4 列布局,或使用窄列来创建间隔。
\figures{fig6-4}{简单多行应用程序的底层结构}
\section{多页面布局}
随着你的应用程序变得越来越复杂,可能无法将所有内容都放在一个页面上。在本节中,你将学习 tabPanel() 创建多个页面错觉的各种用法。这是一种错觉,因为你仍然拥有一个带有单个底层 HTML 文件的应用程序,但它现在被分成了几部分,并且一次只能看到一个部分。

多页面应用程序与模块配合得特别好,你将在 \autoref{ch18} 中了解这些模块。模块允许你以与划分用户界面相同的方式划分服务器功能,创建仅通过明确定义的连接进行交互的独立组件。
\subsection{选项卡集}
将页面分成多个部分的简单方法是使用 tabsetPanel() 以及与它近似的 tabPanel()。如果你想知道用户选择了哪个选项卡,你可以提供参数 id,它将成为 tabsetPanel() 的输入。

注意:tabsetPanel()可以在应用程序中的任何位置使用;如果需要的话,将选项卡集嵌套在其他组件(包括选项卡集!)中是完全可以的。
\subsection{导航列表和导航栏}
由于选项卡是水平显示的,因此可以使用的选项卡数量存在根本限制,特别是当它们具有长标题时。 navbarPage() 与 navbarMenu() 提供两种替代布局,让你可以使用更多带有更长标题的选项卡。

navlistPanel() 类似于 tabsetPanel(),但它不是水平运行选项卡标题,而是在侧边栏中垂直显示它们。它还允许你添加带有纯字符串的标题。另一种方法是使用 navbarPage():它仍然水平运行选项卡标题,但你可以使用 navbarMenu() 添加下拉菜单以获得额外的层次结构级别。

\section{Bootstrap}
要继续您的应用程序定制之旅,您需要更多地了解 Shiny 使用的Bootstrap框架。Bootstrap 是 HTML 约定、CSS 样式和 JS 片段的集合,它们捆绑成一种方便的形式。Bootstrap 源自最初为 Twitter 开发的框架,在过去 10 年中已发展成为网络上最流行的 CSS 框架之一。

作为一个 Shiny 开发者,你不需要对 Bootstrap 考虑太多,因为 Shiny 函数会自动为你生成 bootstrap 兼容的 HTML。但很高兴知道 Bootstrap 的存在,因为这样:

\begin{itemize}
    \item 您可以使用 bslib::bs\_theme() 自定义代码的视觉外观
    \item 您可以使用class参数通过 Bootstrap 类名来自定义一些布局、输入和输出
    \item  您可以创建自己的函数来生成Shiny 未提供的Bootstrap 组件
\end{itemize}

也可以使用完全不同的 CSS 框架。许多现有的 R 包通过包装 Bootstrap 的流行替代品使这一切变得容易:
\begin{itemize}
    \item shiny.semantic,由 Appsilon 开发,建立在 formantic UI 之上。
    \item shinyMobile 由 RInterface 开发,构建于框架 7之上,专为移动应用程序而设计。
    \item shinymaterial 由 Eric Anderson开发,建立在 Google 的 Material 设计框架之上。
    \item shinydashboard 也由 RStudio 提供,提供了一个旨在创建仪表板的布局系统。
\end{itemize}

您可以在 \href{https://github.com/nanxstats/awesome-shiny-extensions}{awesome-shiny-extensions} 找到更完整且积极维护的列表。
\section{主题}
Bootstrap 在 R 社区中如此普遍,以至于很容易产生风格疲劳:一段时间后,每个 Shiny 应用程序和 Rmd 开始看起来都一样。解决方案是使用 bslib 包进行主题化。bslib 是相对较新的软件包,它允许您覆盖许多 Bootstrap 默认值,以创建独特的外观。
\subsection{shiny 主题}
更改应用程序整体外观的最简单方法是使用 的参数选择预制的 “\href{https://bootswatch.com/}{bootswatch}” 主题。
\figures{fig6-9}{同一个应用程序具有四个 bootswatch 主题:darkly、flatly、sandstone 和 United}

预览和自定义主题的一种简单方法是使用 \verb|bslib::bs_theme_preview(theme)|。 这将打开一个闪亮的应用程序,显示应用许多标准控件时主题的外观,并为您提供用于自定义最重要参数的交互式控件。

\subsection{情节主题}
如果您对应用程序的样式进行了大量自定义,您可能还需要自定义绘图以匹配。幸运的是,这真的很容易,这要归功于 thematic 包,它自动为 ggplot2、lattice和 base 提供主题。只需在您的服务器调用 \verb|thematic_shiny()| 功能即可。这将自动确定应用程序主题的所有设置。