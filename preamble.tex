\usepackage{amsmath, amssymb}
\usepackage{ulem}
% \usepackage{mathptmx}
\usepackage{siunitx}
\usepackage{ctex}
\usepackage{minted}
\usepackage{tcolorbox}
\usepackage{epigraph}
\usepackage{caption}
\usepackage{subcaption}
\usepackage{pdfpages}
\usepackage{graphicx}
\setkeys{Gin}{width=0.7\textwidth}
\usepackage{listings}
\newtheorem{theorem}{Theorem}
\usepackage{tikz}
\usepackage{pifont}
\usepackage{tabularx}
\usepackage{framed}
% \usepackage{algorithm}
\usepackage[lined,boxed,ruled]{algorithm2e}
\usepackage{titlesec}
\usepackage{bm}

\titlespacing\section{0pt}{8pt plus 4pt minus 2pt}{0pt plus 2pt minus 2pt}
\titlespacing\subsection{0pt}{8pt plus 4pt minus 2pt}{0pt plus 2pt minus 2pt}
\titlespacing\subsubsection{0pt}{8pt plus 4pt minus 2pt}{0pt plus 2pt minus 2pt}

% 在part页添加内容
\makeatletter
\let\old@endpart\@endpart
\renewcommand\@endpart[1][]{%
    \begin{quote}#1\end{quote}%
    \old@endpart}
\makeatother

% \setminted[python]{bgcolor=red!5, xleftmargin=20pt, linenos, breakanywhere=true,fontsize=\small,baselinestretch=1.1}
\setminted[python]{breakanywhere=true,fontsize=\small,baselinestretch=1.1}
%定义新的minted命令
\newminted[pyc]{python}{}



\newcommand\tips[1]{\textcolor{green!50!black}{#1}}
\newcommand\notes[1]{\textcolor{blue!50!black}{#1}}
\newcommand\important[1]{\textcolor{red!90!black}{#1}}

\newcommand\figures[1]{
    \begin{figure}
        \centering
        \includegraphics{../Figures/#1.png}
        \caption{#1}
        \label{#1}
    \end{figure}
}

\usepackage{enumitem}
% 去掉enumerate、itemize、description中的间隙
\setlist{noitemsep, topsep=0pt}

\tcbuselibrary{minted, skins, breakable}
\tcbset{breakable, colframe=orange!90!black,colback=orange!10, sharp corners=all}
% \newtcblisting[auto counter, number within =chapter]{py}[1]{listing engine=minted,
% 	minted style=colorful,
% 	minted language=python,
% 	minted options={breaklines,autogobble,linenos,numbersep=3mm},
% 	colback=red!5!white,colframe=orange!50!blue,listing only, left=5mm,enhanced,
% 	title=Examples~\thetcbcounter~#1,
% 	breakable,
% 	enhanced
% 	%						 overlay={
% 	%						 	\begin{tcbclipinterior}
% 	%						 		\fill[red!30!white] (frame.south west)
% 	%						 		rectangle ([xshift=5mm]frame.north west);
% 	%							\end{tcbclipinterior}}
% }
% \usepackage[
%     bottom=1.77cm,
%     left=2.75cm,
%     right=2.75cm,
%     marginparwidth=1.5cm,     % + <- Width of your marginpar
%     marginparsep=5mm,       % + <- Gap between text block and marginpar
% ]{geometry}

\usepackage{hyperref}
% 将引用的chapter改写为Chapter
\usepackage[english]{babel}
\addto\extrasenglish{
    \def\chapterautorefname{Chapter}
}
\addto\extrasenglish{
    \def\sectionautorefname{Section}
}
\usepackage{orcidlink}
\definecolor{SWJTU}{HTML}{025483}
% 全局取消段前缩进
% \setlength{\parindent}{0pt}
% \setlength{\parskip}{5pt}
\hypersetup{
    colorlinks=true,
    linkcolor=SWJTU,
    filecolor=SWJTU,
    urlcolor=SWJTU,
    citecolor=SWJTU,
}


