\chapter{数据转换\label{ch03}}
\section{引言}
tibble 是 tidyverse 使用的一种特殊类型的数据框,以避免一些常见的问题。tibbles 和数据框之间最重要的区别是 tibbles 的打印方式。它们是为大型数据集而设计的,因此它们仅显示前几行和适合一个屏幕的列。有几个选项可以查看所有内容。如果你使用 RStudio,最方便的可能是 \verb|View(flights)|,它将打开交互式可滚动和可过滤视图。否则,你可以使用 \verb|print(flights, width = Inf)| 显示所有列,或使用 \verb|glimpse()|。
\subsection{dplyr 基础知识}
学习主要的 dplyr 动词(函数),这将使你能够解决绝大多数数据操作挑战。但在我们讨论它们的个体差异之前,有必要先说明一下它们的共同点:
\begin{itemize}
    \item 第一个参数始终是数据框。
    \item 后续参数通常使用变量名称(不带引号)描述要操作的列。
    \item 输出始终是新的数据框。
\end{itemize}

dplyr 的动词根据其操作内容分为四组:rows、columns、groups 以及 table。
\section{行}
对数据集的行进行操作的最重要的动词是 filter(),它改变存在的行而不改变它们的顺序,以及 arrange() ,它改变存在的行的顺序而不改变存在的行。这两个函数都只影响行,列保持不变。我们还将讨论 distinct(),用于查找具有唯一值。但与 arrange() 和 filter() 不同的是它还可以选择修改列。
\subsection{filter()}
filter() 允许你根据列的值保留行\footnote{稍后,你将了解系列 slice\_*(),它允许你根据位置选择行。}。第一个参数是数据框,第二个和后续参数是保留该行必须满足的条件。

当使用 $|$ 和 == 时有一个有用的快捷方式:\verb|%in%|。它保留变量等于右侧值之一的行。

当你运行 filter() 时,会执行过滤操作,创建一个新的数据框,然后打印它。它不会修改现有数据集,因为 dplyr 函数永远不会修改其输入。要保存结果,你需要使用赋值运算符。
\subsection{arrange()}
arrange() 根据列的值更改行的顺序。它需要一个数据框和一组列名(或更复杂的表达式)来排序。如果你提供多个列名称,则每个附加列将用于打破前一列值中的联系。

在 arrange() 内部可以对某列使用 desc() 然后来根据该列按降序(从大到小)顺序对数据框重新排序。
\subsection{distinct()}
distinct() 查找数据集中的所有唯一行,因此从技术意义上来说,它主要对行进行操作。然而,大多数时候,你需要某些变量的不同组合,因此你还可以选择提供列名称。

如果你想在过滤唯一行时保留其他列,则可以使用该 \verb|.keep_all = TRUE| 选项。distinct() 将找到数据集中第一次出现的唯一行并丢弃其余行。

如果你想查找出现的次数,最好替换 distinct() 为 count(),并且使用 \verb|sort = TRUE| 参数可以按出现次数的降序排列它们。

\section{列}
有四个重要的动词会影响列而不更改行:mutate() 创建从现有列派生的新列、select() 更改存在的列、rename() 更改列的名称以及 relocate() 更改列的位置。

\subsection{mutate()}
mutate() 的工作是添加根据现有列计算的新列。默认情况下,mutate() 在数据集的右侧添加新列,这使得很难看到此处发生的情况。我们可以使用 \verb|.before| 参数将变量添加到左侧。

该 . 符号表明 .before 是函数的参数,而不是我们正在创建的第三个新变量的名称。你还可以 .after 在变量后面添加,并且在两者中 .before 都 .after 可以使用变量名称而不是位置。

或者,你可以使用 .keep 参数控制哪些变量与参数保留。一个特别有用的选项是 "used" 来指定我们只保留步骤中涉及或创建的列 mutate()。

\subsection{select()}
获取包含数百甚至数千个变量的数据集并不罕见。在这种情况下,第一个挑战通常只是关注你感兴趣的变量。select() 允许你使用基于变量名称的操作快速放大有用的子集。

通过 =, 你可以 select() 使用来重命名变量。新名称出现在 = 的左侧,旧变量出现在右侧。

\subsection{rename()}
如果你想保留所有现有变量并且只想重命名一些变量,你可以使用 rename() 替换 select()。

如果你有一堆命名不一致的列,并且手动修复它们会很痛苦,请检查 \href{https://sfirke.github.io/janitor/reference/clean_names.html}{janitor::clean\_names()} 它是否提供了一些有用的自动清理。
\subsection{relocate()}
relocate() 用于移动变量。你可能希望将相关变量收集在一起或将重要变量移到前面。默认情况下将变量移到前面。 你还可以使用 .before 和 .after 参数指定放置它们的位置,与 mutate() 类似。
\section{管道}
\section{组}
当你添加与群组合作的功能时,dplyr 会变得更加强大。在本节中,我们将重点关注最重要的函数:\verb|group_by()|、\verb|summarize()| 和 slice 函数系列。
\subsection{group\_by()}
\verb|group_by()| 用于将数据集分为对分析有意义的组。group\_by() 不会更改数据,但是,如果你仔细查看输出,你会注意到输出表明它是按分组变量“分组”的。这意味着后续操作现在将按组进行。group\_by() 将此分组功能(称为类)添加到数据框中,这会更改应用于数据的后续动词的行为。

\subsection{summarize()}
最重要的分组操作是摘要,如果用于计算单个摘要统计量,则会将数据框减少为每个组只有一行\footnote{summarise(),如果你更喜欢英式英语。}。

你可以在一次调用中创建任意数量的 summarize()。有各种有用的摘要,但一个非常有用的摘要是 n(),它返回每组中的行数。
\subsection{slice\_ 函数}
有五个方便的函数允许你提取每个组中的特定行:
\begin{itemize}
    \item df $|>$ slice\_head(n = 1) 取出每组的第一行。
    \item df $|>$ slice\_tail(n = 1) 占据每组的最后一行。
    \item df $|>$ slice\_min(x, n = 1) 取列 x 具有最小值的行。
    \item df $|>$ slice\_max(x, n = 1) 取列 x 具有最大列值的行。
    \item df $|>$ slice\_sample(n = 1) 随机取一行。
\end{itemize}

你可以 \verb|n| 选择多行也可以使用 \verb|prop = 0.1| 选择(例如)每组中 10\% 的行。

\subsection{按多个变量分组}
你可以使用多个变量创建组。当你对按多个变量分组的小标题进行汇总时,每个汇总都会剥离最后一组。事后看来,这并不是让这个函数发挥作用的好方法,但是在不破坏现有代码的情况下很难进行更改。为了让发生的情况一目了然,dplyr 会显示一条消息,告诉你如何更改此行为。

如果您对此行为感到满意,您可以明确请求它以抑制该消息:\verb|.groups = "drop_last"|,或者,通过设置不同的值来更改默认行为,例如,"drop" 删除所有分组或 "keep" 保留相同的组。

\begin{tcolorbox}
    这里主要是说 grouped\_df 类型的对象是有一个 groups 值会自动删除最后一个组。
\end{tcolorbox}
\subsection{取消分组}
您可能还想从数据框中删除分组而没有使用 summarize(). 您可以使用 来执行此操作 ungroup()。
\subsection{.by}
dplyr 1.1.0 包含用于按操作分组的新的实验性语法,即参数 .by。\verb|group_by()| 并且 \verb|ungroup()| 不会消失,但您现在还可以使用 .by 参数在单个操作中进行分组。

.by 适用于所有动词,其优点是您不需要使用参数 .groups 来抑制分组消息或者分组后再使用 \verb|ungroup()|。

\section{案例研究:总量和样本量}
\href{https://www.evanmiller.org/how-not-to-sort-by-average-rating.html}{How Not To Sort By Average Rating}

\href{https://github.com/JPL-JUNO/Data-Analysis-and-Science/blob/main/R4DS/code/ch02/case_study.R}{警告:仅按均值排序可能会导致错误}