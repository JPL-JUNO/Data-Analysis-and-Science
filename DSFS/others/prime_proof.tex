\documentclass{minimal}
\usepackage{amsmath, amsthm}
\newtheorem{theorem}{定理}
\renewcommand{\proofname}{\indent\textbf{证明}}
\usepackage{ctex}
\begin{document}
\begin{theorem}[素数判定]
    每个整数$n\geq 2$或是素数或是一些素数的乘积。
\end{theorem}

\begin{proof}
    假设结论不成立,则存在“反例”,即一定存在整数 $n\geq 2$ 既不是素数也不是一些素数的乘积. 根据最小反例,可令 $m$ 是这些整数中最小的一个,因为 $m$ 不是素数,则 $m$ 是合数,因此存在因子分解 $m=ab$, $2 \leq a < m$,$2 \leq b < m$(因为 a 是整数,所以由 $1<a$ 可知 $2\leq a$.

    因为 m 是最小反例,所有 a 和 b 都使定理成立,即
    $$a = pp^{\prime}p^{\prime\prime}\cdots,~b = qq^{\prime}q^{\prime\prime}\cdots$$
    其中因子 $p, p^{\prime}, p^{\prime\prime}$ 和 $q, q^{\prime}, q^{\prime\prime}$ 都是素数,因此

    $$m = ab = pp^{\prime}p^{\prime\prime}\cdots qq^{\prime}q^{\prime\prime}\cdots$$
    是一些(至少两个)素数的乘积,矛盾.
\end{proof}
Stephen CUI 写于 2023 年 08 月 18 日
\end{document}