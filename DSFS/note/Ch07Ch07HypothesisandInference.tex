\chapter{Hypothesis and Inference}
\section{Example: Flipping a Coin}
We are also often interested in the \emph{power}(势) of a test, which is the probability of not
making a \emph{type 2 error} (“false negative”), in which we fail to reject $H_0$ even though it's
false.
\section{p-Values}
An alternative way of thinking about the preceding test involves p-values. Instead of
choosing bounds based on some probability cutoff, we compute the probability—
assuming $H_0$ is true—that we would see a value at least as extreme as the one we
actually observed.

\begin{tcolorbox}
    Why did we use a value of 529.5 rather than using 530? This is
    what's called a \href{https://en.wikipedia.org/wiki/Continuity_correction}{continuity correction}. It reflects the fact that nor
    \verb|mal_probability_between(529.5, 530.5, mu_0, sigma_0)| is a
    better estimate of the probability of seeing 530 heads than nor
    \verb|mal_probability_between(530, 531, mu_0, sigma_0)| is.
\end{tcolorbox}
\section{Confidence Intervals}
We've been testing hypotheses about the value of the heads probability p, which is a
parameter of the unknown “heads” distribution. When this is the case, a third
approach is to construct a \emph{confidence interval} around the observed value of the
parameter.

\section{p-Hacking}
\section{Example: Running an A/B Test}
假设:\\
One of
your advertisers has developed a new energy drink targeted at data scientists, and the
VP of Advertisements wants your help choosing between advertisement A (“tastes
great!”) and advertisement B (“less bias!”).


Let's say that $N_A$ people see ad A, and that $n_A$ of them click it. We can think of each ad
view as a Bernoulli trial where $p_A$ is the probability that someone clicks ad A. Then (if
$N_A$ is large, which it is here) we know that $n_A/N_A$ is approximately a normal random
variable with mean $p_A$ and standard deviation $\sigma_A =
    \sqrt{p_A(1 - p_A)/N_A}$.

Similarly, $n_B/N_B$ is approximately a normal random variable with mean $p_B$ and standard deviation $\sigma_B =
    \sqrt{p_B(1 - p_B)/N_B}$.

If we assume those two normals are independent (which seems reasonable, since the
individual Bernoulli trials ought to be), then their difference should also be normal
with mean $p_B - p_A$ and standard deviation $\sigma_A^2 + \sigma_B^2$.
\begin{tcolorbox}
    This is sort of cheating. The math only works out exactly like this if
    you know the standard deviations. Here we're estimating them
    from the data, which means that we really should be using a t-distribution. But for large enough datasets, it's close enough that it
    doesn't make much of a difference.
\end{tcolorbox}

\section{Bayesian Inference}
An alternative approach to inference involves treating the unknown parameters
themselves as random variables. The analyst (that’s you) starts with a \emph{prior distribution} for the parameters and then uses the observed data and Bayes’s theorem to get an
updated \emph{posterior distribution} for the parameters. Rather than making probability
judgments about the tests, you make probability judgments about the parameters.