\chapter{Statistics}
\section{Describing a Single Set of Data}
One obvious description of any dataset is simply the data itself. For a small enough dataset, this might even be the best description. But for
a larger dataset, this is unwieldy and probably opaque. (Imagine staring at a
list of 1 million numbers.) For that reason, we use statistics to distill(提取) and
communicate relevant features of our data.

As a first approach, you put the data counts into a histogram using
Counter and plt.bar (\autoref{A histogram of friend counts}).

\figures{A histogram of friend counts}

\subsection{Central Tendencies}
Usually, we'll want some notion of where our data is centered. Most
commonly we'll use the mean (or average), which is just the sum of the
data divided by its count.

We'll also sometimes be interested in the median, which is the middle-most
value (if the number of data points is odd) or the average of the two middle-
most values (if the number of data points is even).

Notice that—unlike the mean—the median doesn't fully depend on every
value in your data. For example, if you make the largest point larger (or the
smallest point smaller), the middle points remain unchanged, which means
so does the median.

At the same time, the mean is very sensitive to outliers in our data. I If outliers are likely to be
bad data (or otherwise unrepresentative of whatever phenomenon we're
trying to understand), then the mean can sometimes give us a misleading
picture.

A generalization of the median is the quantile, which represents the value
under which a certain percentile of the data lies (the median represents the
value under which 50\% of the data lies).

Less commonly you might want to look at the mode, or most common
value(s).

\subsection{Dispersion(离散度)}
Dispersion refers to measures of how spread out our data is. Typically
they're statistics for which values near zero signify not spread out at all and
for which large values (whatever that means) signify very spread out. For
instance, a very simple measure is the range, which is just the difference
between the largest and smallest elements.

A more complex measure of dispersion is the variance.

Now, whatever units our data is in, all of our measures of
central tendency are in that same unit. The range will similarly be in that
same unit. The variance, on the other hand, has units that are the square of
the original units. As it can be hard to make sense
of these, we often look instead at the standard deviation.

Both the range and the standard deviation have the same outlier problem
that we saw earlier for the mean. A more robust alternative computes the difference between the 75th
percentile value and the 25th percentile value, which is quite plainly unaffected by a small number of outliers.

\section{Correlation}
We'll look at covariance, the paired analogue of variance. Whereas
variance measures how a single variable deviates from its mean, covariance
measures how two variables vary in tandem from their means.

Nonetheless, this number can be hard to interpret, for a couple of reasons:
\begin{enumerate}
    \item Its units are the product of the inputs' units, which can be hard to make sense of.
    \item If one data had twice as many friends (but the same number of
          the other), the covariance would be twice as large. But in a sense,
          the variables would be just as interrelated. Said differently, it's hard
          to say what counts as a “large” covariance.
\end{enumerate}

For this reason, it's more common to look at the correlation, which divides
out the standard deviations of both variables.

The correlation is unitless and always lies between –1 (perfect
anticorrelation) and 1 (perfect correlation). Correlation can be very sensitive to outliers.

\section{Simpson's Paradox}
One not uncommon surprise when analyzing data is Simpson's paradox, in which correlations can be misleading when confounding variables are ignored.
\important{The key
    issue is that correlation is measuring the relationship between your two
    variables all else being equal.} If your dataclasses are assigned at random, as
they might be in a well-designed experiment, “all else being equal” might
not be a terrible assumption. But when there is a deeper pattern to class
assignments, “all else being equal” can be an awful assumption.


The only real way to avoid this is by knowing your data and by doing what
you can to make sure you've checked for possible confounding factors.
Obviously, this is not always possible. 但是有些时候没有必要的数据,可能得出的结论就会有问题。
\section{Some Other Correlational Caveats}
A correlation of zero indicates that there is no linear relationship between
the two variables. However, there may be other sorts of relationships.

In addition, correlation tells you nothing about how large the relationship is. 因为有些强关系,但是实际中没有任何的意义。

\section{Correlation and Causation}
You have probably heard at some point that “correlation is not causation.”. Nonetheless, this is an
important point—if x and y are strongly correlated, that might mean that x
causes y, that y causes x, that each causes the other, that some third factor
causes both, or nothing at all.

One way to feel more confident about causality is by conducting
randomized trials. If you can randomly split your users into two groups with
similar demographics and give one of the groups a slightly different
experience, then you can often feel pretty good that the different
experiences are causing the different outcomes.