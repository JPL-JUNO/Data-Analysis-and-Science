\chapter{Probability}
\section{Continuous Distributions}

\section{The Normal Distribution}
The normal distribution is the classic bell curve–shaped distribution and is com‐
pletely determined by two parameters: its mean μ (mu) and its standard deviation σ
(sigma). The mean indicates where the bell is centered, and the standard deviation
how “wide” it is.

It has the PDF:
\begin{equation*}
    f(x|\mu, \sigma)=\frac{1}{\sqrt{2\pi}\sigma}\exp(-
    \frac{(x-\mu)^2}{2\sigma^2})
\end{equation*}

When$\mu = 0$ and $\sigma = 1$, it’s called the standard normal distribution. If $Z$ is a standard
normal random variable, then it turns out that:
$$X = \sigma Z + \mu$$
is also normal but with mean $\mu$ and standard deviation $\sigma$. Conversely, if X is a normal
random variable with mean $\mu$ and standard deviation $\sigma$,
$$Z = X - \mu /\sigma$$
is a standard normal variable.
The CDF for the normal distribution cannot be written in an “elementary” manner,
but we can write it using Python’s \verb|math.erf| \href{https://en.wikipedia.org/wiki/Error_function}{error function}: